\usepackage[dvipsnames]{xcolor}
% \usepackage[altpo]{backnaur}

\usepackage{
  amsmath, amssymb, amsthm, thmtools, thm-restate, tikz, graphicx,
  hyperref, cleveref, comment, expl3, xparse, ebproof, enumitem, stmaryrd,
  enumitem, verbatim, todonotes, xargs, url, mathtools, mathrsfs, xfrac, cite,
  setspace, titlesec, bbm, dirtytalk, fancyhdr, esvect, tikz, listings,
  % simplebnf,
  naive-ebnf
}

% \RenewDocumentCommand{\bnfexpr}{m}{$\langle \texttt{#1} \rangle$}

% \newcommand{\nonterminal}[1]{
%   $\langle \text{#1} \rangle$
% } 

\lstset{basicstyle=\small\ttfamily}

\crefname{lstlisting}{listing}{listings}
\Crefname{lstlisting}{Listing}{Listings}

\usetikzlibrary{shapes.geometric, arrows}

\newcommandx{\mygather}[3][1=1ex, 2=\normalsize]{
  \begin{spreadlines}{#1}
    {#2
      \begin{gather*}
        #3
      \end{gather*}
    }%
  \end{spreadlines}
}

\newcommandx{\myenumerate}[2][1=1ex]{
  \begin{enumerate}
    \setlength \itemsep{#1}
    #2
  \end{enumerate}
}

\newcommandx{\myitemize}[2][1=1ex]{
  \begin{itemize}
    \setlength \itemsep{#1}
    #2
  \end{itemize}
}

\newcommandx{\myalign}[3][1=1ex, 2=\normalsize]{
  \begin{spreadlines}{#1}
    {#2
    \begin{align*}
      #3
    \end{align*}
    }%
  \end{spreadlines}
}

\DeclareMathOperator{\Boolean}{\mathtt{Boolean}}
\DeclareMathOperator{\Class}{\mathtt{Class}}
\DeclareMathOperator{\Number}{\mathtt{Number}}
\DeclareMathOperator{\Object}{\mathtt{Object}}

\DeclareMathOperator{\decl}{decl}

\DeclareMathOperator{\Pre}{Pre}
\DeclareMathOperator{\Post}{Post}

\DeclareMathOperator{\true}{\mathtt{true}}
\DeclareMathOperator{\false}{\mathtt{false}}

\DeclareMathOperator{\dom}{dom}
\DeclareMathOperator{\N}{\mathbb{N}}

\newcommandx{\set}[2]{
  \{#1 \, | \, #2\}
}

\newcommandx{\mcrl}{\texttt{mcrl2}}

\newcommandx{\class}[2]{
    \texttt{class } #1 \, \{ #2 \}
}

\newcommandx{\classextends}[3]{
    \texttt{class } #1 \texttt{ extends } #2 \, \{ #3 \}
}

\newcommandx{\pair}[2]{
    (#1, \, #2)
}

\newcommandx{\jdgmt}[3][1=\Gamma, 2=\Delta]{
    #1; \, #2 \vdash #3
}

\newcommandx{\subtypejdgmt}[2]{
   #1 \leq_\Delta #2 
}

% https://tex.stackexchange.com/questions/9796/how-to-add-todo-notes
\newcommandx{\info}[2][1=] {
  \vspace{5mm}
  \todo[inline, linecolor=OliveGreen,backgroundcolor=OliveGreen!25,bordercolor=OliveGreen,#1]{#2}
  \vspace{5mm}
}

\newcommandx{\inlinetodo}[1]{
  \info{
    \textbf{TODO}

    #1
  }
}