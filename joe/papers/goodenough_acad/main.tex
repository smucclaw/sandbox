\documentclass{article}
\usepackage[utf8]{inputenc}

\usepackage[dvipsnames]{xcolor}
% \usepackage[altpo]{backnaur}

\usepackage{
  amsmath, amssymb, amsthm, thmtools, thm-restate, tikz, graphicx,
  hyperref, cleveref, comment, expl3, xparse, ebproof, enumitem, stmaryrd,
  enumitem, verbatim, todonotes, xargs, url, mathtools, mathrsfs, xfrac, cite,
  setspace, titlesec, bbm, dirtytalk, fancyhdr, esvect, tikz, listings,
  % simplebnf,
  naive-ebnf
}

% \RenewDocumentCommand{\bnfexpr}{m}{$\langle \texttt{#1} \rangle$}

% \newcommand{\nonterminal}[1]{
%   $\langle \text{#1} \rangle$
% } 

\lstset{basicstyle=\small\ttfamily}

\crefname{lstlisting}{listing}{listings}
\Crefname{lstlisting}{Listing}{Listings}

\usetikzlibrary{shapes.geometric, arrows}

\newcommandx{\mygather}[3][1=1ex, 2=\normalsize]{
  \begin{spreadlines}{#1}
    {#2
      \begin{gather*}
        #3
      \end{gather*}
    }%
  \end{spreadlines}
}

\newcommandx{\myenumerate}[2][1=1ex]{
  \begin{enumerate}
    \setlength \itemsep{#1}
    #2
  \end{enumerate}
}

\newcommandx{\myitemize}[2][1=1ex]{
  \begin{itemize}
    \setlength \itemsep{#1}
    #2
  \end{itemize}
}

\newcommandx{\myalign}[3][1=1ex, 2=\normalsize]{
  \begin{spreadlines}{#1}
    {#2
    \begin{align*}
      #3
    \end{align*}
    }%
  \end{spreadlines}
}

\DeclareMathOperator{\Boolean}{\mathtt{Boolean}}
\DeclareMathOperator{\Class}{\mathtt{Class}}
\DeclareMathOperator{\Number}{\mathtt{Number}}
\DeclareMathOperator{\Object}{\mathtt{Object}}

\DeclareMathOperator{\decl}{decl}

\DeclareMathOperator{\Pre}{Pre}
\DeclareMathOperator{\Post}{Post}

\DeclareMathOperator{\true}{\mathtt{true}}
\DeclareMathOperator{\false}{\mathtt{false}}

\DeclareMathOperator{\dom}{dom}
\DeclareMathOperator{\N}{\mathbb{N}}

\newcommandx{\set}[2]{
  \{#1 \, | \, #2\}
}

\newcommandx{\mcrl}{\texttt{mcrl2}}

\newcommandx{\class}[2]{
    \texttt{class } #1 \, \{ #2 \}
}

\newcommandx{\classextends}[3]{
    \texttt{class } #1 \texttt{ extends } #2 \, \{ #3 \}
}

\newcommandx{\pair}[2]{
    (#1, \, #2)
}

\newcommandx{\jdgmt}[3][1=\Gamma, 2=\Delta]{
    #1; \, #2 \vdash #3
}

\newcommandx{\subtypejdgmt}[2]{
   #1 \leq_\Delta #2 
}

% https://tex.stackexchange.com/questions/9796/how-to-add-todo-notes
\newcommandx{\info}[2][1=] {
  \vspace{5mm}
  \todo[inline, linecolor=OliveGreen,backgroundcolor=OliveGreen!25,bordercolor=OliveGreen,#1]{#2}
  \vspace{5mm}
}

\newcommandx{\inlinetodo}[1]{
  \info{
    \textbf{TODO}

    #1
  }
}

% Title information
\title{Formalizing a simple loan agreement in L4 and Maude}
\author{Joe Watt}
% \date{June 2022}

\begin{document}

% Title and table of contents.
\maketitle

\info{
  This is a work in progress.
}

\tableofcontents

\newpage

% Main body.
\section{Intro and shortcomings of DFA formalization}

\subsection{Intro}

% \inlinetodo{
%   Write a better intro and provide more background on
%   \cite{contract_as_automaton}.
% }
\inlinetodo{
  Write a proper introduction to the original work on formalizing the simple
  loan agreement as a DFA.
}

% In \cite{contract_as_automaton}, Flood and Goodenough propose that many financial
% contracts are inherently computational in nature.
% They argue that the computational structure of many such contracts
% can be formalized via deterministic finite automata (DFAs), with states
% representing various situations.
% % like ``borrower default'' and ``payment due''.
% Transitions between these states then correspond to events triggering a change
% in these situations.
% This is demonstrated using a simple loan agreement
% \cite[Table 1]{contract_as_automaton}, which they formalize
% directly as a DFA.

In \cite{contract_as_automaton}, Flood and Goodenough propose that many
financial contracts are inherently computational in nature.
They argue that the computational structure of many such contracts can be
formalized via deterministic finite automata (DFAs), with states representing
various situations.
Transitions between these states then correspond to events triggering a change
in these situations.
This is demonstrated using a simple loan agreement \cite{contract_as_automaton},
which they formalize directly as a DFA.
% Finally, they claim that such a formalism opens up the possibility of performing
% automated analyses on contracts.

\subsection{Shortcomings}
While this approach is a useful and important proof of concept, as the authors
themselves recognize, there are some limitations with such a formalism, perhaps
the most apparent being that the manual encoding of a contract as a DFA is a
laborious process, and one that in itself does not produce a machine executable
result
One source of complexity arises from the fact that the DFA is a low level
computational formalism that is arguably far removed from our intuitive
understanding of a contract.
While we can easily encode real-world events as transitions in a DFA, as done
in \cite{contract_as_automaton}, some legal traditions reason about and specify
contracts in terms of normative requirements that need to be fulfilled by
various actors \cite{normative_compliance_guido}.
The authors of this paper, representing different traditions, do not entirely
agree on this point, but it remains true that a DFA is a relatively
mechanistic chain of event and consequence, and is not adept at normative
representation.

Contracts also frequently embody some notion of time and deadline.
These are examples of domain-specific concepts which are an integral
part of contract drafting.
Unfortunately, DFAs have no primitive notion of these, so that these must all
be encoded manually, with the passage of time reflected as an achieved event,
in a manner that is perhaps unnatural to many.

Another source of complexity is the state explosion problem.
Complex contracts have large DFA representations and are thus impractical for
humans to specify manually at scale.
Observe that a simplified visual representation of the automaton
\cite[Fig. 1]{contract_as_automaton}
corresponding to this simple contract already contains more than 20 states and
40 transitions.

% Consequently, a more practical formalism should provide mechanisms like
% global variables and arithmetic operations.

% Such an issue could be addressed by a formalism that is more sophisticated than
% a DFA, something that allows for global variables, which can be updated at
% runtime.

% Ideally, we would like to be able to collapse these into a single subgraph
% corresponding to the``accelerated repayments'' phase of the contract.
% However, this is not possible here as both accelerated repayment stages as in
% the 

% Likewise, the transitions between them are similar as well, with the only
% difference being the 

% % verbose duplication of states along the main ``happy path''.

% What we mean instead is that there are similarly named states whose subgraphs
% have similar structure.
% These include the 2 main repayment stages along this path, given by the
% ``Payment\dots accruing'' states.
% Each of the subgraphs rooted at these nodes has a similar structure, with
% borrower default events giving rise to transitions to the unhappy paths.
% These in turn also have similar structure, with similar
% ``Payment\dots accelerating'' and ``Crisis\dots'' states.
% The only difference between the transitions between these states is

One of the main causes of this is the concurrent interleaving of real
world events.
By this, we mean that many real world events can occur in any order (including
at the same time) and
perhaps for the sake of simplicity, this has not been accounted for in the DFA
formalization.
As an example of such a scenario, consider the following.
On May 31, 2015, the day before payment 1 is due, the borrower defaults on his
representations and warranties.
On the next day, when payment 1 becomes due, the borrower diligently
repays that payment.
One day later, on June 2, 2015, the lender notifies the borrower of his
earlier default, which does not get cured after another 2 days.
Now all outstanding payments become accelerated, and the borrower pays off
the remaining amount of \$525 in time, causing the contract to terminate.
This sequence of events, when viewed as a word over the event alphabet, is
unfortunately not accepted by the automaton.

Closer inspection of the DFA suggests that there is an
implicit assumption being made that once an event of default occurs,
the borrower will be notified by the lender soon after, with no other events
like payments occurring in between.

More formally, it is assumed that the default and notification events occur
within the same atomic step.
Splitting this up into separate events would increase the number of states in
the DFA, especially if we account for concurrency.
In fairness, Flood and Goodenough understood some of these limits, and specified
in \cite{contract_as_automaton}[Section 6] of their model agreement that:
``In the event of multiple events of default, the first to occur shall take
precedence for the purposes of specifying outcomes under this agreement''.
This provision goes some way toward resolving competing defaults by the
expedient of mandating that later events will simply be disregarded.

% Workspace
% \info{
%   Other weaknesses to mention:

%   Representation of time and deadlines.
%   Not handled well by the DFA, but is an important notion that exists in L4.
%   These are translated to Maude.

%   No explicit notion of deontics, which is important to humans because that's
%   what people are used to reasoning about and normative language shows up
%   everywhere in legal drafting.
%   In L4, we have a notion of deontics, though not quite the same as deontic logic.
%   We have an operational interpretation that is made precise by the translation
%   to Maude.

%   The focus here is that a friendly-to-use language should provide appropriate
%   abstractions corresponding to concepts found in the application domain.
%   For the legal domain, these include the notions of rules, actors, events, 
%   deadlines and deontics.
%   Another reason for modelling these explicitly is to be able to achieve a
%   nice correspondence (more technically an isomorphism) with legal texts that
%   are drafted in natural language.
% }

% Old not useful
% It should be noted here that this only poses problems in the manual encoding
% of a contract, by hand, as a DFA.
% For the purposes of automated reasoning as in the field of 
% \textit{formal verification}, model checking tools like UPPAAL
% \todo{cite properly here}
% are able to automatically analyze automata comprising thousands of states.

% Therefore, we argue that this evinces that accurately
% encoding contracts as a DFA directly is too laborious and difficult,
% even for a contract as simple as the one presented in
% \cite[Fig 1.]{contract_as_automaton}.
% In our view, a more practical formalism should sit at a higher-level than a DFA,
% providing mechanisms for tackling these 2 issues.
% Firstly, it should provide global variables, or at least, some way of encoding
% global state.
% There should also be operations that allow us to retrieve and update the global
% state.
% Secondly, it should be able to conveniently accommodate the concurrency
% inherent in the real world.
% Consequently, we believe that a practical formalism for encoding contracts
% should be something higher-level than a DFA, one that can conveniently
% accommodate the concurrency inherent in the real world.
% Maudeetter still if the formalism comes with tools that allow us to compile down
% to something like automata for visualization and automated reasoning.

% However, it should be noted that for the purposes of visualization and
% automated reasoning in the sense of \textit{formal verification}, DFAs are a
% suitable choice of formalism.
% Indeed, tools such as UPPAAL \todo{list more tools and cite properly} are able to
% efficiently simulate and analyze automata with thousands of states.
% In fact, these tools rely on analyzing \textit{labelled transition systems},
% generalizations of nondeterministic finite automata (NFA) that allow for
% infinitely \footnote{Possibly uncountably} many states and transitions.

\section{The L4 approach}

\subsection{Intro to L4 as a language}
At CCLAW, we are developing one such domain specific language (DSL) for
specifying legal contracts, called L4, which addresses these issues.
Along with it, we are also developing an accompanying toolset which can be used
to visualize and analyze contracts written in L4.

\inlinetodo{
  Talk about constitutive and regulative rules.

  Constitutive rules as statics, semantics is arguably straightforward, namely
  some first order defeasible logic.
  Can be formalized using Prolog.

  Regulative rules as dynamics, trickier semantics, is the focus of this paper.
  Will show how the loan agreement is modelled using these.

  Give an example of a regulative rule found in the loan agreement to
  demonstrate the syntax. Then use this example to highlight domain specific
  concepts.
  Read \cite{normative_diags_diogo,real_time_contract_automata} 
  and highlight similarities with them.
}

As mentioned earlier, we believe that a friendly to use language for specifying
legal contracts should be situated at a higher level of abstraction than that
of automata, closer to the application domain.

\info{
  Workspace:

  It comes with a concurrent model of computation (this is made precise
  by the translation to Maude) which allows for arbitrary interleaving of events.

  It is also friendly to use, containing ontological
  concepts from the legal domain that people are used to thinking about and
  working with.

  Discuss the precise roles of these concepts and how the execution of a
  contract can be viewed abstractly as a process that arises from the interaction
  between them.

  Mention that the precise operational interpretation of rules in L4 is tightly
  linked to the translation in Maude and this was heavily inspired by
  \cite{symboleo_model_checking_contracts,eventb_data_sharing_agreements}.

  We view a rule as a wrapper around an event, a deontic (ie permission,
  obligation or prohibition) and an actor.
  Our treatment of deontics accounts for the notion of \textit{compensability},
  and is based on \cite{normative_compliance_guido}.

  Note that the difference is that instead of reasoning about deontics using
  deontic logic, we provide an \textit{operational} interpretation for them.
}

\subsection{The loan agreement in L4}

% \subsection{From L4 to Maude}
\section{L4 through Maude}

In search for such a suitable formalism for encoding contracts,
we have surveyed various approaches, using the simplified contract in
\cite[Fig 1.]{contract_as_automaton} as an example.
One of these which we explored is the Maude language and its associated tools.

In this section, we begin by providing some background on Maude before
we discuss our formalization of the simple loan agreement.
Along the way, we explain how we addressed the aforementioned shortcomings.
In particular, we show how we can generate an arguably more accurate DFA that
accounts for the concurrent interleaving of real world events.
Thereafter, we demonstrate how we can use Maude to help us
visualize and reason about the contract.

\subsection{Background on Maude}
Techniques originating from the field of formal methods have been devised to
automatically analyze the behavior of computer systems.
These often rely on first modelling these systems as
\textit{labelled transition systems} (LTS), which can be seen as
generalizations of nondeterministic finite automata).
As with finite automata, LTSes can be seen as directed graphs, with nodes
representing states that the system can be in, and labelled edges denoting
transitions that change the state of a system.
Where they differ from finite automata is that they are allowed to have an
infinite (possibly uncountable) number of states and transitions
between them.
They are also not required to come with a notion of initial and final states.

While DFAs and LTSes in general are formalisms that are well suited for
computers to analyze and reason about, they are cumbersome for humans to use
to encode systems.
This is especially so for systems like the simple loan agreement of
\cite{contract_as_automaton} which involve concurrency, in that there are
many possible ways in which events can be interleaved.
Modelling concurrent systems like this directly as a transition system is
often impractical as the interleaving of events gives rise to numerous states
and transitions.
Moreover, as previously discussed, they leave implicit many of the domain
specific concepts that people directly reason about.

Maude is a language that allows us to model and analyze such concurrent systems.
As with others like it, it comes equipped with a more sophisticated formalism
that allow us to more conveniently specify these transition systems.

\subsection{Translating from L4 to Maude}
Use the loan agreement as an example here.

\subsection{Visualizing and simulating the contract}
Insert DFA of state space here.

\subsection{Analyzing via model checking}
Here we should mention that in \cite{temporal_logic_norms_guido}, the author
highlights an issue with using temporal logic to reason about deontics like
obligations.
We sidestep that issue due to the way we define rules as wrappers around
events, and deontics, so that the violation of a rule corresponds directly to
whether that event occured or not.

For instance, a rule of the form ``Party P must do A by D'' is an obligation
that is violated if P does not perform A by the deadline D.
In this way, we can reason about the violation of obligations by using
temporal logic to talk about whether an event occurred or not.

\section{Related work}

\section{Limitations and future work}
The current Maude models are not very performant and have huge state spaces.
One of the main reasons is that we model time naively, with a \texttt{tick}
transition denoting the passage of 1 unit of time (which can be taken to be
1 day, 1 month etc).
We did this because Maude does not come with any built-in support for these notions,
so we wanted an easy way to model these.
Unfortunately, this considerably bloats up the state space.
Again it must be emphasized that this is a proof of concept and these were
designed to be as high-level and close to the syntax of L4 as possible.
In the future, we intend to optimize this, among other things.
% We could investigate the use of other formalisms that better support modeling
% time and deadlines, or we could build our own abstractions to handle these more
% efficiently in Maude.

Currently, we don't yet support global variables and more sophisticated control
structures like loops.
Global variables would be useful to capture the notion of the outstanding amount
in the loan agreement contract, which would then vary with the payments made by
the borrower.
Maude as a formalism supports these, but we currently do not support them in L4.
For future work, we would like to provide syntax for these in L4 and translate
them to Maude.

\newpage

% Maudeibliography stuff.
\bibliographystyle{acm}
\bibliography{refs}

\end{document}