\documentclass{article}
\usepackage[utf8]{inputenc}

\usepackage[dvipsnames]{xcolor}
% \usepackage[altpo]{backnaur}

\usepackage{
  amsmath, amssymb, amsthm, thmtools, thm-restate, tikz, graphicx,
  hyperref, cleveref, comment, expl3, xparse, ebproof, enumitem, stmaryrd,
  enumitem, verbatim, todonotes, xargs, url, mathtools, mathrsfs, xfrac, cite,
  setspace, titlesec, bbm, dirtytalk, fancyhdr, esvect, tikz, listings,
  % simplebnf,
  naive-ebnf
}

% \RenewDocumentCommand{\bnfexpr}{m}{$\langle \texttt{#1} \rangle$}

% \newcommand{\nonterminal}[1]{
%   $\langle \text{#1} \rangle$
% } 

\lstset{basicstyle=\small\ttfamily}

\crefname{lstlisting}{listing}{listings}
\Crefname{lstlisting}{Listing}{Listings}

\usetikzlibrary{shapes.geometric, arrows}

\newcommandx{\mygather}[3][1=1ex, 2=\normalsize]{
  \begin{spreadlines}{#1}
    {#2
      \begin{gather*}
        #3
      \end{gather*}
    }%
  \end{spreadlines}
}

\newcommandx{\myenumerate}[2][1=1ex]{
  \begin{enumerate}
    \setlength \itemsep{#1}
    #2
  \end{enumerate}
}

\newcommandx{\myitemize}[2][1=1ex]{
  \begin{itemize}
    \setlength \itemsep{#1}
    #2
  \end{itemize}
}

\newcommandx{\myalign}[3][1=1ex, 2=\normalsize]{
  \begin{spreadlines}{#1}
    {#2
    \begin{align*}
      #3
    \end{align*}
    }%
  \end{spreadlines}
}

\DeclareMathOperator{\Boolean}{\mathtt{Boolean}}
\DeclareMathOperator{\Class}{\mathtt{Class}}
\DeclareMathOperator{\Number}{\mathtt{Number}}
\DeclareMathOperator{\Object}{\mathtt{Object}}

\DeclareMathOperator{\decl}{decl}

\DeclareMathOperator{\Pre}{Pre}
\DeclareMathOperator{\Post}{Post}

\DeclareMathOperator{\true}{\mathtt{true}}
\DeclareMathOperator{\false}{\mathtt{false}}

\DeclareMathOperator{\dom}{dom}
\DeclareMathOperator{\N}{\mathbb{N}}

\newcommandx{\set}[2]{
  \{#1 \, | \, #2\}
}

\newcommandx{\mcrl}{\texttt{mcrl2}}

\newcommandx{\class}[2]{
    \texttt{class } #1 \, \{ #2 \}
}

\newcommandx{\classextends}[3]{
    \texttt{class } #1 \texttt{ extends } #2 \, \{ #3 \}
}

\newcommandx{\pair}[2]{
    (#1, \, #2)
}

\newcommandx{\jdgmt}[3][1=\Gamma, 2=\Delta]{
    #1; \, #2 \vdash #3
}

\newcommandx{\subtypejdgmt}[2]{
   #1 \leq_\Delta #2 
}

% https://tex.stackexchange.com/questions/9796/how-to-add-todo-notes
\newcommandx{\info}[2][1=] {
  \vspace{5mm}
  \todo[inline, linecolor=OliveGreen,backgroundcolor=OliveGreen!25,bordercolor=OliveGreen,#1]{#2}
  \vspace{5mm}
}

\newcommandx{\inlinetodo}[1]{
  \info{
    \textbf{TODO}

    #1
  }
}

% Title information
\title{Formalizing a simple loan agreement in mcrl2}
\author{Joe Watt}
% \date{June 2022}

\begin{document}

% Title and table of contents.
\maketitle

\info{
  This is a work in progress.
}

\tableofcontents

% Main body.
\section{Intro and shortcomings of DFA formalization}

\inlinetodo{
  Write a better intro and provide more background on
  \cite{contract_as_automaton}.
}

In \cite{contract_as_automaton}, the authors claim that many financial
contracts are inherently computational in nature.
They argue that the computational structure of many such contracts
can be formalized via deterministic finite automata (DFAs), with states
representing various situations.
% like ``borrower default'' and ``payment due''.
Transitions between these states then correspond to events triggering a change
in these situations.
This is demonstrated using a simple loan agreement
\cite[Table 1]{contract_as_automaton}, which they formalize
directly as a DFA.
% Finally, they claim that such a formalism opens up the possibility of performing
% automated analyses on contracts.

While this approach is a nice proof of concept, there are various shortcomings
with such a formalism, one of them being that encoding such a DFA manually
is a laborious process.
A simplified visual representation of the automaton corresponding to this
simple contract \cite[Fig. 1]{contract_as_automaton} already contains more than
20 states and 40 transitions.
This would explode in complexity if the contract were more complex, like say if
it contained more repayment stages instead of just two.

Upon closer inspection of the states in the DFA, one may notice that this
high complexity is a result of a verbose duplication of states along the main
``happy path''.
Note here that when we say ``duplication'', we do not mean that there are states
which play exactly the same role from the point of view of bisimilarity or
language acceptance as in the sense of the Myhill-Nerode technique of DFA
minimization.
What we mean instead is that there are similarly named states whose subgraphs
have similar structure.
These include the 2 main repayment stages along this path, given by the
``Payment\dots accruing'' states.
Each of the subgraphs rooted at these nodes has a similar structure, with
borrower default events giving rise to transitions to the unhappy paths.
These in turn also have similar structure, with similar
``Payment\dots accelerating'' and ``Crisis\dots'' states.
% The only difference between the transitions between these states is

The DFA formalization in \cite{contract_as_automaton} also has some subtle
concurrency bugs in that the DFA does not allow for some possible interleaving
of events.
To see this, consider the following scenario.
On May 31, 2015, the day before payment 1 is due, the borrower defaults on his
representations and warranties.
On the next day, when payment 1 becomes due, the borrower diligently
repays that payment.
One day later, on June 2, 2015, the lender notifies the borrower of his
earlier default, which does not get cured after another 2 days.
Now all outstanding payments become accelerated, and the borrower pays off
the remaining amount of \$525 in time, causing the contract to terminate.
This scenario, when formalized as a word over the event alphabet, is
unfortunately not accepted by the automaton.

Perhaps for the sake of simplicity or otherwise, there is an implicit
assumption being made that once an event of default occurs,
the borrower will immediately be notified by the lender, so that no other
events like payments can occur in between.
A more realistic encoding of this simplified contract should allow for such
interleaving of events, as the real world is inherently concurrent in nature.
However, this would further increase the complexity
of the DFA and thus the tedium involved in its manual encoding.
This suggests that for practical purposes, one should not use a DFA
directly, but rather, a formalism that can conveniently accommodate the
concurrency that is present in the real world.
Better still if it could then be compiled automatically into an automaton
of some kind and automated analysis performed on it.

In search for such a formalism, we have surveyed various approaches for
representing legal contracts, using the simplified contract in
\cite[Fig 1.]{contract_as_automaton} as an example.
In particular, one of the formalisms we explored is the \mcrl \, toolset
\cite{mcrl_book, mcrl_toolset}.
This toolset provides a high-level modeling language based on a
\textit{process algebra}, a formalism \dots
\inlinetodo{Describe process algebras}

Here, we discuss our formalization and show how it generates a DFA that better
handles the concurrent interleaving of events.
We also demonstrate how we can use the \mcrl \, toolset to perform automated
analysis with the contract.

% which make such an encoding cumbersome.

% For instance, there is a verbose duplication of states along the main
% ``happy path'' in \cite[Fig. 1]{contract_as_automaton}.
% Note here that when we say ``duplication'', we do not mean that there are states
% which play exactly the same role from the point of view of language acceptance,
% as in the sense of the Myhill-Nerode technique of DFA minimization.
% What we mean instead is that there are similarly named states whose subgraphs
% have similar structure.


% Beneath these, there are similarly named states representing the occurrence of a
% borrower default event, which in turn lead to payment accelerating states,
% lying along the unhappy paths.
% Here, one sees that the duplicated states lying along the happy path also gives
% rise to a duplication in states along the unhappy paths.

% ``borrower defaults on representations/warranties'' and ``payment made''.

% Bibliography stuff.
\bibliographystyle{apalike}
\bibliography{refs}

\end{document}