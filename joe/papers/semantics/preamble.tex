\usepackage[dvipsnames]{xcolor}
\usepackage[altpo]{backnaur}

\usepackage{
  amsmath, amssymb, amsthm, thmtools, thm-restate, tikz, graphicx,
  hyperref, cleveref, comment, expl3, xparse, ebproof, enumitem, stmaryrd,
  enumitem, verbatim, todonotes, xargs, url, mathtools, mathrsfs, xfrac, cite,
  setspace, titlesec, bbm, dirtytalk, fancyhdr, esvect, tikz, listings
}

\usepackage[mathscr]{euscript}

\lstset{basicstyle=\small\ttfamily}

\crefname{lstlisting}{listing}{listings}
\Crefname{lstlisting}{Listing}{Listings}

\usetikzlibrary{shapes.geometric, arrows}

\newcommandx{\mygather}[3][1=1ex, 2=\normalsize]{
  \begin{spreadlines}{#1}
    {#2
      \begin{gather*}
        #3
      \end{gather*}
    }%
  \end{spreadlines}
}

\newcommandx{\myenumerate}[2][1=1ex]{
  \begin{enumerate}
    \setlength \itemsep{#1}
    #2
  \end{enumerate}
}

\newcommandx{\myitemize}[2][1=1ex]{
  \begin{itemize}
    \setlength \itemsep{#1}
    #2
  \end{itemize}
}

\newcommandx{\myalign}[3][1=1ex, 2=\normalsize]{
  \begin{spreadlines}{#1}
    {#2
    \begin{align*}
      #3
    \end{align*}
    }%
  \end{spreadlines}
}


% https://tex.stackexchange.com/questions/37797/theorem-environment-line-break-after-label
\newtheoremstyle{break}% name
{}%         Space above, empty = `usual value'
{}%         Space below
{\itshape}% Body font
{}%         Indent amount (empty = no indent, \parindent = para indent)
{\bfseries}% Thm head font
{.}%        Punctuation after thm head
{\newline}% Space after thm head: \newline = linebreak
{}%         Thm head spec

% https://tex.stackexchange.com/questions/250035/transform-output-theoremstyleremark-from-italics-to-bold
\newtheoremstyle{boldremark}
{\dimexpr\topsep/2\relax} % space above
{\dimexpr\topsep/2\relax} % space below
{}          % body font
{}          % indent amount
{\itshape\bfseries}% theorem head font
{.}         % punctuation after theorem head
{\newline}      % space after theorem head
{}          % theorem hed spec. (empty = "normal")

\newtheoremstyle{discussion}% name
{}%         Space above, empty = `usual value'
{}%         Space below
{}% Body font
{}%         Indent amount (empty = no indent, \parindent = para indent)
{\bfseries}% Thm head font
{.}%        Punctuation after thm head
{\newline}% Space after thm head: \newline = linebreak
{}%         Thm head spec

\theoremstyle{definition}
\newtheorem{definition}{Definition}[section]
\newtheorem{notation}{Notation}[section]
\newtheorem{example}{Example}[section]
\newtheorem{convention}{Convention}[section]

\theoremstyle{break}
\newtheorem{theorem}{Theorem}[section]
\Crefname{theorem}{Theorem}{Theorems}

\theoremstyle{break}
\newtheorem{corollary}{Corollary}[theorem]
\Crefname{corollary}{Corollary}{Corollaries}

\theoremstyle{break}
\newtheorem{lemma}[theorem]{Lemma}
\Crefname{lemma}{Lemma}{Lemmas}

\theoremstyle{remark}
\newtheorem*{remark}{Remark}

% https://latex.org/forum/viewtopic.php?t=24225
\theoremstyle{discussion}
\newtheorem{discussion}[theorem]{Discussion}
\Crefname{discussion}{Discussion}{Discussions}

\theoremstyle{boldremark}
\newtheorem*{boldremark}{Remark}

% \theoremstyle{convention}
% \newtheorem{convention}[theorem]{Convention}
% \Crefname{convention}{Convention}{Conventions}

% \newcommand{\powerset}{\mathscr{P}}
% \newcommand{\powerset}{\mathscr{P}}
\newcommand{\N}{\mathbb{N}}
\newcommand{\Z}{\mathbb{Z}}
\newcommand{\R}{\mathbb{R}}
\newcommand{\T}{\mathbb{T}}

\DeclareMathOperator{\Boolean}{\mathtt{Boolean}}
\DeclareMathOperator{\Class}{\mathtt{Class}}
\DeclareMathOperator{\Number}{\mathtt{Number}}
\DeclareMathOperator{\Object}{\mathtt{Object}}

\DeclareMathOperator{\decl}{decl}

\DeclareMathOperator{\Pre}{Pre}
\DeclareMathOperator{\Post}{Post}

\DeclareMathOperator{\true}{\mathtt{true}}
\DeclareMathOperator{\false}{\mathtt{false}}

\newcommandx{\set}[1]{
  \{ #1 \}
}

\newcommandx{\setcompre}[2]{
  \{#1 \, | \, #2\}
}

\newcommandx{\pair}[2]{
    (#1, \, #2)
}

\newcommandx{\jdgmt}[3][1=\Gamma, 2=\Delta]{
    #1; \, #2 \vdash #3
}

\newcommand{\curlyA}{\mathscr{A}}

\newcommand{\defeq}{\stackrel{\text{def}}{=}}
\newcommand{\traces}{\mathbf{Tr}}
\newcommand{\fintraces}{\mathbf{Tr}^\text{fin}}

\newcommand{\instwff}{\varphi_\text{inst}}
\newcommand{\suswff}{\varphi_\text{sus}}
\newcommand{\termwff}{\varphi_\text{term}}
\newcommand{\reswff}{\varphi_\text{res}}
\newcommand{\contwff}{\varphi_\text{cont}}
\newcommand{\brewff}{\varphi_\text{bre}}
\newcommand{\fulwff}{\varphi_\text{ful}}

\newcommand{\susstate}{\text{Suspended}}
\newcommand{\expstate}{\text{Expired}}
\newcommand{\ineffstate}{\text{InEffect}}
\newcommand{\brestate}{\text{Breached}}
\newcommand{\fulstate}{\text{Fulfilled}}

\newcommand{\obaut}{\curlyA_O}

% https://tex.stackexchange.com/questions/9796/how-to-add-todo-notes
\newcommandx{\info}[2][1=] {
  \vspace{5mm}
  \todo[inline, linecolor=OliveGreen,backgroundcolor=OliveGreen!25,bordercolor=OliveGreen,#1]{#2}
  \vspace{5mm}
}

\newcommandx{\inlinetodo}[1]{
  \info{
    \textbf{TODO}

    #1
  }
}