% Theorems and definitions

% \newtheorem{definition}{Definition}
% \newtheorem{theorem}{Theorem}
% \newtheorem{lemma}{Lemma}
% \newtheorem{proposition}{Proposition}


% Definition of colors
\newcommand{\blue}[1]{{\color{blue}#1}}
\newcommand{\green}[1]{{\color{green}#1}}
\newcommand{\red}[1]{{\color{red}#1}}
\newcommand{\gray}[1]{{\color{gray}#1}}

% From MSCS file
\newcommand{\eg}{\textit{e.g.\ }}
\newcommand{\etal}{\textit{et al.\ }}
\newcommand{\etc}{\textit{etc}}
\newcommand{\ie}{\textit{i.e.\ }}
\newcommand{\viz}{\textit{viz.\ }}
\newcommand{\wrt}{\textit{w.r.t.\ }}
\newcommand{\lex}{\langle}
\newcommand{\rex}{\rangle}

% Own abbreviations
\newcommand{\secref}[1]{Section~\ref{#1}}
\newcommand{\secrefs}[1]{Sections~\ref{#1}}
\newcommand{\figref}[1]{Figure~\ref{#1}}
\newcommand{\figrefs}[1]{Figures~\ref{#1}}
\newcommand{\pgref}[1]{page~\pageref{#1}}
\newcommand{\theoremref}[1]{Theorem~\ref{#1}}
\newcommand{\theoremrefs}[1]{Theorems~\ref{#1}}
\newcommand{\lemmaref}[1]{Lemma~\ref{#1}}
\newcommand{\exampleref}[1]{Example~\ref{#1}}
\newcommand{\defref}[1]{Definition~\ref{#1}}

\newcommand{\figline}{\rule{\textwidth}{0.5pt}}


% Logique

\newcommand{\IMPL}[0]{\longrightarrow}
\newcommand{\IMPLL}[0]{\Longrightarrow} % another implication, to make
                                % a difference with reduction relations
\newcommand{\AND}[0]{\land}
\newcommand{\OR}[0]{\lor}
\newcommand{\NOT}[0]{\lnot}
\newcommand{\FALSE}[0]{\perp}
\newcommand{\TRUE}[0]{\top}
\newcommand{\IFF}[0]{\leftrightarrow}
\newcommand{\BIGAND}[1]{\bigwedge_{#1}}
\newcommand{\BIGOR}[1]{\bigvee_{#1}}
\newcommand{\BIGANDC}[2]{\bigwedge_{#1|#2}} % bigand with constraint
\newcommand{\BIGORC}[2]{\bigvee_{#1|#2}}    % bigor with constraint

\newcommand{\exgeq}[1]{\exists^{{\geq #1}}}
\newcommand{\exeq}[1]{\exists^{{= #1}}}
\newcommand{\exle}[1]{\exists^{{< #1}}}

% Other

\newcommand{\smalltalcq}[0]{{\small small}-t{$\cal ALCQ$}}
\newcommand{\smalltalcqe}[0]{{\small small}-t{$\cal ALCQ$e}}
\newcommand{\trule}[0]{\xhookrightarrow}
\newcommand{\tableaurule}[1]{{\xhookrightarrow[]{#1}}}
\newcommand{\nodes}[1]{{\cal N}({#1})}
\newcommand{\trans}[1]{{\cal T}({#1})}
\newcommand{\transm}[1]{{\cal T'}({#1})}
\newcommand{\rconts}[1]{\llparenthesis #1 \rrparenthesis} %record contents
\newcommand{\rupd}[2]{{#1}\llparenthesis #2 \rrparenthesis} %record update

\newcommand{\eform}[0]{\mathit{eform}}
\newcommand{\form}[0]{\mathit{form}}
\newcommand{\free}[0]{\mathit{free}}
\newcommand{\exclprop}[0]{\stackrel{\times}{\longrightarrow}}


%----------------------------------------------------------------------
% For drawing syntax diagrams
% ----------------------------------------------------------------------

% Environment defining the general layout

\newenvironment{syntaxdiagram}[1]
{
%  \begin{equation}\label{eq:#1}
  \begin{tikzpicture}[%
node distance=5mm and 10mm,
>=stealth',
black!50,
text=black,
graphs/every graph/.style={edges=rounded corners},
hv path/.style={to path={-| (\tikztotarget)}},
vh path/.style={to path={|- (\tikztotarget)}},
nonterminal/.style={%
rectangle,
minimum size=6mm,
draw=black,
},
terminal/.style={%
rectangle,minimum size=6mm,rounded corners=3mm,
draw=black!50,
},
start/.style={%
circle,inner sep=1pt,minimum size=1pt,fill=white,draw=black!50,
},
end/.style={%
start,
},
junction/.style={circle,inner sep=0pt,minimum size=0pt},]
\node[nonterminal] (#1) {\hypertarget{syn:#1}{#1:}};
}
{\end{tikzpicture}
%\end{equation}
}

% Connects start point #1 via intermediate node entry #2 and exit #3
% to an end point #4.
\newcommand{\syndiagAlternative}[4]{%
\graph[use existing nodes] {
#1 ->[vh path] #2;
#3 ->[hv path] #4;
};
}

% Connects start point #1 via intermediate node #2 (typically a junction)
% to an end point #3.
\newcommand{\syndiagBridge}[3]{%
\graph[use existing nodes] {
#1 --[vh path] #2;
#2 ->[hv path] #3;
};
}

\newcommand{\nonterminalref}[1]{\hyperlink{syn:#1}{#1}}


%----------------------------------------------------------------------
% Remarks
% ----------------------------------------------------------------------

\newcommand{\remms}[2][]{\todo[color=green!40,#1]{MS: #2}}



%%% Local Variables: 
%%% mode: latex
%%% TeX-master: "main"
%%% End: 
