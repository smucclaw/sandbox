\documentclass[english]{article}

%%%%%%%%% PACKAGES %%%%%%%%%%%%%%%%%
\usepackage[utf8]{inputenc}
\usepackage[T1]{fontenc}
\usepackage{amsmath}
\usepackage{amssymb}
\usepackage{amsfonts}
\usepackage{alltt}
\renewcommand{\ttdefault}{txtt} % to resolve a problem with bold fonts in alltt

\usepackage{listings}
\lstset{language=haskell,basicstyle=\ttfamily}

\usepackage{proof}
\inferLineSkip=4pt  % increase spacing between lines; default is 2pt

\usepackage[table]{xcolor}
\usepackage[colorlinks,hyperindex,bookmarks,linkcolor=blue,citecolor=blue,urlcolor=blue]{hyperref}
\usepackage{todonotes}

\usepackage{fancyhdr}

%\usepackage{floatflt}
\usepackage{wrapfig}
\usepackage{caption}
\usepackage{subcaption}
\usepackage{framed}
\usepackage{multicol}

\makeatletter

\usepackage{babel}
\usepackage{graphicx}
\usepackage{theorem}

\usepackage{tikz}
\usetikzlibrary{trees}
\usetikzlibrary{positioning} 
\usepackage{tikzsymbols}


\makeatother

%%%%%%%%% GEOMETRY %%%%%%%%%%%%%%%%%
\addtolength{\topmargin}{-15mm}
\addtolength{\textheight}{25mm}
\addtolength{\oddsidemargin}{-20mm}
\setlength{\textwidth}{16cm}

%%%%%%%%% DECLS / DEFNS %%%%%%%%%%%%%%%%%

\usepackage{comment}
\specialcomment{solution}
{\todo[inline]{BEGIN SOLUTION}}
{\todo[inline]{END SOLUTION}}

{\theorembodyfont{\rmfamily} 
  \newtheorem{exo}{Exercise}
  \newtheorem{rem}{Remark}
}

% Macros for references
\newcommand{\polyref}[1]{polycopié, {#1}}

%%%%%%%%% END DECLS / DEFNS %%%%%%%%%%%%%%%%%

%%% Local Variables: 
%%% mode: latex
%%% coding: utf-8-unix
%%% End: 

%======================================================================
%%%%%%%%% DECLS / DEFNS %%%%%%%%%%%%%%%%%

% redefine existing green color
\newcommand{\brightgreen}[1]{{\color{green}#1}}
\definecolor{darkgreen}{rgb}{0,0.7,0}
\newcommand{\green}[1]{{\color{darkgreen}#1}}

\newcommand{\blue}[1]{{\color{blue}#1}}
%\newcommand{\green}[1]{{\color{green}#1}}
\newcommand{\red}[1]{{\color{red}#1}}
\newcommand{\yellow}[1]{{\color{yellow}#1}}
\definecolor{grey}{rgb}{0.5,0.5,0.5}
\newcommand{\grey}[1]{{\color{grey}#1}}
\newcommand{\black}[1]{{\color{black}#1}}

% Logique
\newcommand{\IMPL}[0]{\longrightarrow}
\newcommand{\IMPLL}[0]{\Longrightarrow} % another implication, to make
                                % a difference with reduction relations
\newcommand{\AND}[0]{\land}
\newcommand{\OR}[0]{\lor}
\newcommand{\NOT}[0]{\lnot}
\newcommand{\FALSE}[0]{\perp}
\newcommand{\TRUE}[0]{\top}
\newcommand{\IFF}[0]{\leftrightarrow}
\newcommand{\BIGAND}[1]{\bigwedge_{#1}}
\newcommand{\BIGOR}[1]{\bigvee_{#1}}
\newcommand{\BIGANDC}[2]{\bigwedge_{#1|#2}} % bigand with constraint
\newcommand{\BIGORC}[2]{\bigvee_{#1|#2}}    % bigor with constraint
\newcommand{\ufb}[0]{\mbox{$\stackrel{?}{=}$}} % unifiable

% semantique
\newcommand{\semtrans}[3]{\mbox{$\langle #1, #2 \rangle \rightarrow #3$}}
\newcommand{\statesel}[2]{\mbox{$#1 . #2$}}
\newcommand{\stateupd}[3]{\mbox{$#1 . #2 \leftarrow #3$}}


% Macros for mathematical notation

\def\N{{\mathbf N}}                      % ensemble des entiers naturels
\def\Z{{\mathbf Z}}                      % ensemble des entiers relatifs
\newcommand{\zmod}[1]{\mbox{$ \Z/{#1} \Z$}}  % anneau Z mod m


% WP-calcul
\newcommand{\wpform}[1]{\mbox{\{\( #1 \)\} }}
\newcommand{\subst}[3]{ #1 [ #2 \mbox{$\leftarrow\ $} #3]}
\newcommand{\pfp}[2]{\mbox{\emph{pfp}(\texttt{#1}, \mbox{$#2$})}}
\newcommand{\evb}[1]{\emph{évaluable}($#1$)}

% special references
\newcommand{\secref}[1]{\S~\ref{#1}}
\newcommand{\transpref}[1]{transparent, p.~\ref{#1}}


\newtheorem{theorem}{Théorème}
\newtheorem{lemma}{Lemme}
\newtheorem{definition}{Définition}
\newtheorem{logic}{Principe}


%%% Local Variables: 
%%% mode: latex
%%% TeX-master: "main"
%%% coding: utf-8
%%% End: 


% Ligne commentée: avec solutions; ligne non commentée: sans solutions
% État par défaut (commit dans le dépôt): décommenté, pour éviter une fuite
% des solutions aux étudiants.
\excludecomment{solution}

\begin{document}

\begin{center}
  {\LARGE Propositional Formulas in Haskell}
\end{center}


%----------------------------------------------------------------------
\section{Context}\label{sec:context}

The following mini-project is meant to provide an illustration for the
data types and pattern matching as used in Haskell. At the same time, it deals with
transformations of (natural or artificial) languages, which is central to the
L4 project.

Specifically, we will look at manipulating propositional formulas in
Haskell. The concept appears under various names, such as Boolean
Algebra\footnote{\url{https://en.wikipedia.org/wiki/Boolean_algebra}} or
Propositional Logic\footnote{same Wikipedia page, or also
  \url{https://en.wikipedia.org/wiki/Propositional_calculus}}. If you have
never in your life heard of them, you should read the introductory sections of
the mentioned Wikipedia pages, but no in-depth knowledge of logic is required
in the following.

Manipulating languages in a computer is not intrinsically difficult, but it
can be bewilderingly confusing because you deal with at least two
languages: the one used for programming (Haskell in our case) and the
language(s) you manipulate (you could for example translate Java to C). To
make things worse, the same language can appear in both roles (you can write a
Haskell compiler in Haskell).

To keep the two levels separate, we will use the following terminology:
\begin{itemize}
\item We will talk about \emph{Boolean expressions} when referring to the
  language features present in the Haskell programming language. Examples are
  \texttt{True \&\& False} or also \texttt{(a \&\& not b) || c}.
\item We will talk about \emph{Boolean formulas} when referring to the
  language elements that we want to represent and manipulate in Haskell. In
  print, we will use conventional mathematical notation: $\TRUE$ for ``true'',
  $\FALSE$ for ``false'', $\NOT$ for ``not'', $\AND$ for ``and'', $\OR$ for
  ``or''. Written as formulas, the above two expressions become $\TRUE \AND
  \FALSE$ and $(a \AND \NOT b) \OR c$. Further below, we will develop an
  internal representation for Boolean formulas.
\end{itemize}



%----------------------------------------------------------------------
\section{Representing Boolean Formulas}

Boolean formulas can represent simple factual statements, and we want to
reason about these statements and, in particular, know whether they are
``true'' (this notion is more complex than it seems, see \secref{sec:evaluating_overview}).
For example, $\TRUE \AND \FALSE$ is always false. To confirm it, you can type the
corresponding expression into Haskell:

\begin{lstlisting}
> True && False
False
\end{lstlisting}

Similarly, $r \AND \NOT r$ is always false - if $r$ means ``it rains'', it is
clear that it cannot rain and not rain at the same time. Now, type the
expression \texttt{r \&\& not r} into Haskell and see what happens - certainly
not what you want.

We will now conceive a representation of formulas in Haskell. Under this
representation, $\TRUE \AND \FALSE$ will be \texttt{(C True) `And` (C False)}
and $r \AND \NOT r$ will be \texttt{V "r" `And` (Not (V "r"))}.\footnote{The
  word \texttt{And} is surrounded by back-quotes (slanted rightwards) and not
  straight quotes. When copy-pasting from a PDF document, you sometimes have
  to modify them manually.} After definition of the corresponding data type,
you can type these expressions into Haskell, and nothing evil happens. To say
the truth, nothing happens at all - we will have to tell Haskell what to do
with these formulas, see \secref{sec:manipulating} and
\secref{sec:evaluating}.

We will first develop a type \texttt{Form} of formulas. It will be very similar
to the \texttt{Expr} type you find in the Haskell book (p. 416). The
\texttt{Form} type has five \emph{constructors}, roughly corresponding to
$\TRUE, \FALSE, \NOT, \AND, \OR$. However, we will combine $\TRUE$ and
$\FALSE$ into the constructor \texttt{C} (for constant), which takes a
\texttt{Bool} as argument, and we also 
have to represent variables, for which we use the constructor \texttt{V},
which takes a string (the variable name) as argument. 

\begin{exo}\label{exo:formula_representation}
Define the data type \texttt{Form}. Make sure that you can type in Haskell
formulas, such as the one
corresponding to $(a \AND \NOT b) \OR c \OR \FALSE$:
\begin{lstlisting}
> ((V "a" `And` (Not (V "b"))) `Or` ((V "c") `Or` (C False)))
Or (And (V "a") (Not (V "b"))) (Or (V "c") (C False))
\end{lstlisting}

\emph{Hint:} your definition should look something like the following, with \texttt{...} to
be filled in:
\begin{lstlisting}
data Form
  = ...
  | Form `Or` Form
  deriving (Eq, Ord, Show, Read)
\end{lstlisting}
\end{exo}
Some further remarks:
\begin{itemize}
\item In Haskell, you can convert a \emph{prefix} function symbol (like
  \texttt{And}, that precedes its arguments) into an \emph{infix} function
  symbol (that is surrounded by its arguments) by including it into
  back-quotes. 
\item \emph{And} and \emph{or} are naturally associative -- you can parenthesise
  them either way: $a \AND b \AND c$ ``is the same as'' $(a \AND b) \AND c$ or
  $a \AND (b \AND c)$. In the internal Haskell representation, we have to be
  more precise and put the parentheses explicitly.
\end{itemize}

Sure enough, it is tedious to have a rather complex Haskell representation for
a formula that looks rather simple in print. Converting a print / textual
representation into the internal Haskell representation is the job of a
\emph{parser}\footnote{We can maybe have a follow-up exercise sheet on
  parsing.}.

\begin{exo}\label{exo:const_var_constructors}
Make sure you understand why it is necessary to wrap Boolean constants into
\texttt{C} and variables into \texttt{V}:
\begin{itemize}
\item Why can't we write \texttt{True `And` False} but have to write
  \texttt{(C True) `And` (C False)}? 
\item Why can't we write \texttt{r `And` (Not r)} but have to write
  \texttt{V "r" `And` (Not (V "r"))}? 
\end{itemize}
Typing the ``wrong'' formulas into the Haskell interpreter might give you an
idea ...
\end{exo}



%----------------------------------------------------------------------
\section{Manipulating Boolean Formulas}\label{sec:manipulating}

%......................................................................
\subsection{Simplifying Constants}\label{sec:simplif_const}

The first formula manipulation will be to simplify expressions by eliminating
constants in formulas. For example, $a \AND \TRUE$ is equivalent to $a$. It is
here necessary to make a careful distinction between two notions of equality:
\begin{itemize}
\item $a \AND \TRUE$ and $a$ are structurally not the same formula
  (\emph{syntactically}, they are different). In Haskell:
  \begin{lstlisting}
> ((V "a") `And` (C True)) == (V "a")
False
  \end{lstlisting}
\item but $a \AND \TRUE$ and $a$ have the same meaning (\emph{semantic
    equivalence}). For equivalence, we also use the symbol $\equiv$.
\end{itemize}

The aim of the following exercises is to eliminate constants $\TRUE$ and
$\FALSE$ from a formula, using equivalences such as $f \AND \TRUE \equiv f$ or
$\FALSE \AND f \equiv \FALSE$. In the end of the process, we will obtain a
formula containing no more constants (unless the formula is constantly $\TRUE$
or $\FALSE$).

We proceed in two stages. Let us first write a function \texttt{removeConst ::
  Form -> Form} that eliminates constants occurring in combination with the
unary operator $\NOT$ or the binary operators $\AND$ and $\OR$.
The clauses corresponding to the above two equivalences are:

\begin{lstlisting}
removeConst (f `And` (C True)) = f
removeConst ((C False) `And` f) = (C False)
\end{lstlisting}
You see that pattern matching comes in handy to enumerate the possible cases.

\begin{exo}\label{exo:removeConst}
Complete the definition of \texttt{removeConst}, by systematically exploring
all possible combinations (2 for $\NOT$, 4 each for $\AND$ and $\OR$). You
will need one final catch-all clause that covers the case where a formula is
neither of the preceding forms:
\begin{lstlisting}
removeConst f = f
\end{lstlisting}
What happens if you put this as the first clause and not the last clause?
Make sure you understand why the ordering of the clauses matters.
\end{exo}

When testing this function, you will see that \texttt{((V "a") `And` ((C
  False) `Or` (C True)))} is not simplified. What is going wrong? 
The problem is that \texttt{removeConst} only simplifies away constants that occur
at the top level but not more deeply inside formulas. For this to happen, we
have to proceed to the second stage of the simplification process, for which
we write function \texttt{simplifyConst :: Form -> Form}. The strategy will be
to recursively simplify sub-formulas (returning a constant-free formula or
$\TRUE$ or $\FALSE$) and then to remove possible constants at the top level
with \texttt{removeConst}. Here is the clause for \texttt{And}:
\begin{lstlisting}
simplifyConst (f1 `And` f2) = removeConst ((simplifyConst f1) `And` (simplifyConst f2))
\end{lstlisting}


\begin{exo}\label{exo:simplifyConst}
Complete the definition of \texttt{simplifyConst}. 
\end{exo}

We now get the expected behaviour:
\begin{lstlisting}
> simplifyConst ((V "a") `And` ((C False) `Or` (C True)))
V "a"
\end{lstlisting}



%......................................................................
\subsection{Normal Forms}\label{sec:normal_forms}

Removing constants as in \secref{sec:simplif_const} is a first step in a
process to derive normal forms that can be processed by more efficient methods
to check the truth of formulas. The following exercises
provide some more (optional) training material on manipulating formulas
without providing essential new insights. If you like, you can directly move to
\secref{sec:evaluating} for a more elementary method for checking the truth of formulas.

\paragraph{Negation normal form (nnf)} The \emph{nnf} of a formula is a form
where all negations $\NOT$ are pushed as far inside a formula as possible. For
computing it, one
repeatedly applies the following equivalences:
\begin{itemize}
\item $\NOT \NOT f \equiv f$
\item $\NOT (f_1 \AND f_2) \equiv (\NOT f_1) \OR (\NOT f_2)$
\item $\NOT (f_1 \OR f_2) \equiv (\NOT f_1) \AND (\NOT f_2)$
\end{itemize}

\begin{exo}\label{exo:nnf}
  Write the function \texttt{nnf :: Form -> Form} that computes the \emph{nnf}
  of a formula. The definition is rather straightforward, but do not forget to
  propagate negations also in nested subformulas, by recursive calls. You may
  assume that constants have already been removed as of
  \secref{sec:simplif_const}, which reduces the number of patterns you have to
  consider.
\end{exo}

\noindent
\emph{Example:}
\begin{lstlisting}
> nnf (Not ((V "a") `And` ((V "b") `Or` (Not (V "a")))))
Not (V "a") `Or` (Not (V "b") `And` V "a")
\end{lstlisting}
As you see, the \emph{nnf} is characterized by the fact that negations can
only occur directly in front of variables.


\paragraph{Conjunctive normal form (cnf)} The \emph{cnf} of a formula is a
``polynomial'' representation of a formula in which all the disjunctions
($\OR$) have been distributed over the conjunctions ($\AND$) so that a formula
is a conjunction of disjunctions of literals (where a literal is a variable or
a negated variable).

For this, one exhaustively applies the following equivalences:
\begin{itemize}
\item $p \OR (q \AND r) \equiv (p \OR q) \AND (p \OR r)$
\item $(q \AND r) \OR p  \equiv (q \OR p) \AND (r \OR p)$
\end{itemize}

\begin{exo}\label{exo:cnf}
Write function \texttt{cnf :: Form -> Form} that computes the \emph{cnf} of a
formula.

\emph{Hint:} it may be useful to define an auxiliary function
\texttt{distribOr :: Form -> Form -> Form} that distributes an \texttt{Or} over an \texttt{And}
according to the above two equations, so that \texttt{distribOr f1 f2} is
equivalent to \texttt{f1 `Or` f2} and such that the result is in \emph{cnf} if
both of \texttt{f1} and \texttt{f2} are. The method is analogous to the
interplay of \texttt{removeConst} and \texttt{simplifyConst} in \secref{sec:simplif_const}.
\end{exo}


\noindent
\emph{Example:}
\begin{lstlisting}
> cnf ((V "a" `And` V "b") `Or` (V "c" `And` V "d"))
((V "a" `Or` V "c") `And` (V "b" `Or` V "c")) `And`
((V "a" `Or` V "d") `And` (V "b" `Or` V "d"))
\end{lstlisting}

Well, these formulas are not very readable. We have converted $(a \AND b) \OR (c \AND d)$ into
$(a \OR c) \AND (b \OR c) \AND (a \OR d) \AND (b \OR d)$ (omitting some
parentheses so that the structure of the \emph{cnf} becomes clearer).


%----------------------------------------------------------------------
\section{Evaluating Boolean Formulas}\label{sec:evaluating}

% ......................................................................
\subsection{Overview}\label{sec:evaluating_overview}

In this section, we will explore how to find out whether a Boolean formula
is ``true''. The method chosen is not the most efficient one, but we can use
previously defined functions such as the ones of \secref{sec:simplif_const}.

We will illustrate the procedure with the formula $a \AND (\NOT a \OR \NOT b)$. It
cannot be simplified in an obvious way, because it contains variables $a$ and
$b$. To find out whether it could be true, let's replace $a$ by $\TRUE$, which
yields $\TRUE \AND (\NOT \TRUE \OR \NOT b)$. We could in principle simplify
this formula, but it still depends on $b$, so we repeat the process (for
example replacing $b$ by $\FALSE$), to obtain $\TRUE \AND (\NOT \TRUE \OR \NOT
\FALSE)$. And indeed, this formula simplifies to $\TRUE$.

Could we say that the formula is ``true''? Certainly if $a=\TRUE, b=\FALSE$, but
not under all circumstances, because for $a=\TRUE, b=\TRUE$, the formula is
equivalent to $\FALSE$. In logic, an assignment of values to variables such as
$a=\TRUE, b=\FALSE$ is called a \emph{valuation}. With this notion, we can
give a more differentiated definition of what it means for a formula to be
``true'':
\begin{itemize}
\item A formula is called \emph{valid} if it is true for every valuation, for
  example $a \OR b \OR \NOT a$ (try it out, by simplifying it with the 4
  possible valuations). 
\item A formula is called \emph{satisfiable} if it is true for at least one
  valuation, for example $a \AND (\NOT a \OR \NOT b)$. A valid formula is a
  special case of a satisfiable formula. Formulas that are satisfiable but not
  valid are sometimes called
  \emph{contingent}\footnote{\url{https://en.wikipedia.org/wiki/Contingency_(philosophy)}}. 
\item A formula is called \emph{unsatisfiable} if it is true for no valuation,
  such as for example $a \AND \NOT a$.
\end{itemize}
A valuation that makes a formula true is also called a \emph{model} of the
formula (so $a=\TRUE, b=\FALSE$ is a model of $a \AND (\NOT a \OR \NOT b)$),
and the process of verifying whether a valuation is a model is called
\emph{model checking}. In the following, we will systematically enumerate all
possible valuations to determine the truth status of a given formula.


% ......................................................................
\subsection{Auxiliary Functions}\label{sec:evaluating_aux}

We  first need to find out what the variables of a formula are.

\begin{exo}\label{exo:fvList}
Write the function  \texttt{fvList :: Form -> [String]} that returns a list of the
variable names of the formula (\texttt{fv} is for ``free variables''). 

\emph{Hint:} The arguably most difficult cases are the binary operators. Take
the example of the formula $a \AND (\NOT a \OR \NOT b)$. A recursive call of
\texttt{fvList} on the left subformula yields \texttt{["a"]}; for the right
subformula, you get \texttt{["a", "b"]}. Concatenating these two lists with
\texttt{++} is a good idea, but you will now have two occurrences of
\texttt{"a"} in the result. To remove duplicates, use the built-in function
\texttt{nub}. For this, you will have to \texttt{import Data.List} at the
beginning of the module.
\end{exo}

\noindent
\emph{Example:}
\begin{lstlisting}
> fv (V "a" `And` (Not (V "a") `Or` (Not (V "b"))))
["a","b"]
\end{lstlisting}


We then need a function to replace a variable by a constant in a formula. This
function will be called \texttt{subst} (for substitution), and it has type
\texttt{Form -> (String, Bool) -> Form}, where the second argument is a
(variable name, value) pair. Alternatively, we could have chosen the type
\texttt{Form -> String -> Bool -> Form}, but the first type is a better fit
for our representation of valuations later on.

\begin{exo}\label{exo:subst}
Code function \texttt{subst}, which is a relatively straightforward traversal
of the formula structure. The only difficult case is the variable case, where
you have to make a case distinction. For
example, \texttt{subst (V "a") ("b", True)} should return the same variable:
\texttt{(V "a")}, whereas \texttt{subst (V "b") ("b", True)} should return
\texttt{C True}. 
\end{exo}

\noindent
\emph{Example:}
\begin{lstlisting}
> subst (V "a" `And` (Not (V "a") `Or` (Not (V "b")))) ("a", True)
C True `And` (Not (C True) `Or` Not (V "b"))
\end{lstlisting}

We will represent a valuation by a list of pairs (variable, value). The
valuation $a=\TRUE, b=\FALSE$ will thus be written as \texttt{[("a", True),
  ("b", False)]}. Whereas \texttt{subst} substituted a single value for a
single variable, the following \texttt{substAll} substitutes a valuation in a
formula.

\begin{exo}\label{sec:substAll}
Code function \texttt{substAll :: Form -> [(String, Bool)] -> Form}. Use
function \texttt{subst} in your definition. You can either make an explicit
recursion over the (variable, value) list; or you can take a very close look
at the predefined function \texttt{foldl} and see how it can be used to define
\texttt{substAll}. In its most cut-down version, the definition is just two words!
\end{exo}

\noindent
\emph{Example:}
\begin{lstlisting}
> substAll (V "a" `And` (Not (V "a") `Or` (Not (V "b")))) [("a", True), ("b", False)]
C True `And` (Not (C True) `Or` Not (C False))
\end{lstlisting}

\begin{exo}\label{exo:evalSubst}
Write function \texttt{evalSubst :: Form -> [(String, Bool)] -> Bool} that
computes the value of a formula under a valuation, assuming that the valuation
assigns a value to all the variables occurring in the formula (this is an
essential condition - without that, the function fails). This is easy to
do: you first substitute the valuation in the formula (with
\texttt{substAll}), then simplify with \texttt{simplifyConst} and finally
retrieve the Boolean value you obtain.
\end{exo}

\noindent
\emph{Example:}
\begin{lstlisting}
> evalSubst (V "a" `And` (Not (V "a") `Or` (Not (V "b")))) [("a", True), ("b", False)]
True
> evalSubst (V "a" `And` (Not (V "a") `Or` (Not (V "b")))) [("a", True), ("b", True)]
False
\end{lstlisting}

% ......................................................................
\subsection{Model Finding}\label{sec:evaluating_models}

We now come to the point of systematically enumerating all valuations to
determine the truth status of a formula. We aim at a function \texttt{allModels}
that returns a list of all models:

\noindent
\emph{Example:}
\begin{lstlisting}
> allModels (V "a" `Or` (Not (V "b")))
[[("b",True),("a",True)],
 [("b",False),("a",True)],
 [("b",False),("a",False)]]
\end{lstlisting}

For this, we will use an auxiliary function \texttt{models} that is
conceptually the most difficult of the functions defined here (even though its
implementation just comprises two lines of code). \texttt{models} takes a
formula, a (partial) valuation and a list of variables still to be processed,
and it returns a list of models.\\
\texttt{models :: Form -> [(String, Bool)] -> [String] -> [[(String, Bool)]]}

As an invariant, we will ensure that the variables of the valuation together
with the variables still to be processed are exactly the variables of the
formula. 

The function is defined by recursion over the third argument,
\emph{i.e.} the list of variables.
\begin{itemize}
\item If the list of variables is empty (thus for the case \texttt{models f vl
    []}), we can evaluate the formula \texttt{f} under the valuation
  \texttt{vl} (using \texttt{evalSubst}). If the formula is true under the
  valuation, we return it as a model (\texttt{[vl]}); otherwise, we return the
  empty list \texttt{[]}. 
\item If there are still variables to be processed (thus for a case of the
  form \texttt{models f vl (vn : vns)}), we will see what happens if the
  variable \texttt{vn} is assigned the value \texttt{True}, and what happens
  it it is assigned the value \texttt{False}. For the \texttt{True} case, we
  will get the models with the call \texttt{(models f ((vn, True):vl) vns)},
  and analogously for the \texttt{False} case. We just have to concatenate the
  two results (with \texttt{++}). 
\end{itemize}

\begin{exo}\label{exo:models}
Hopefully, this explanation is clear enough so that you can flesh out the
definition of \texttt{models}.
\end{exo}

\begin{exo}\label{exo:allModels}
The definition of \texttt{allModels :: Form -> [[(String, Bool)]]} is now
easy: given a formula, we generate the models by calling \texttt{models} with
the list of valuations initially empty and the variables to be processed
initialised to the list of variables of the formula.
\end{exo}

\begin{exo}\label{exo:unsatisfiable}
  It is now clear how to check if a formula is unsatisfiable: generate the
  list of all of its models and check whether the list is empty or
  not. Accordingly, define \texttt{unsatisfiable :: Form -> Bool}.
\end{exo}

\begin{exo}\label{exo:valid}
Validity can be reduced to unsatisfiability: $f$ is valid if and only if $\NOT
f$ is unsatisfiable. With this, the definition of \texttt{valid :: Form ->
  Bool} is a one-liner.
\end{exo}


\noindent
\emph{Examples:}
\begin{lstlisting}
> valid (V "a" `Or` (Not (V "b")))
False
> valid (V "a" `Or` (Not (V "a")))
True
> unsatisfiable (V "a" `Or` (Not (V "b")))
False
> unsatisfiable (V "a" `Or` (Not (V "a")))
False
\end{lstlisting}


%----------------------------------------------------------------------
\section{Propositional Prolog}\label{sec:prolog}

% ......................................................................
\subsection{Informal overview}\label{sec:prolog_overview}

This section is about a very impoverished version of the
Prolog\footnote{\url{https://en.wikipedia.org/wiki/Prolog}} language which we
call Propositional Prolog (PP) because its ``predicates'' have no arguments and are
therefore just propositional variables.

Here is the famous ``Socrates is mortal'' argument as PP program:
\begin{alltt}
  h.
  m :- h.
  ?- m
\end{alltt}
It states that Socrates is a human (\texttt{h.}), that for being mortal, it is
sufficient to be human (\texttt{m :- h.}), and we ask whether Socrates is
mortal (\texttt{?- m}). When running this program, it answers with
\texttt{Yes}.

Some terminology:
\begin{itemize}
\item The first element (\texttt{h.}) is called a \emph{fact}. It makes a
  statement about something assumed to be true.
\item The second element (\texttt{m :- h.}) is called a \emph{rule},
  consisting of a \emph{head} (\texttt{m}) and a \emph{body} (\texttt{h}). It
  says that the head is true provided the body is true.
\item The last element (\texttt{?- m}) is called a \emph{query} by which we
  want to ascertain whether the proposition of the query can be derived from
  the facts and rules.
\end{itemize}
A PP program can be made up of many facts and rules (in any order), but only
one query.

A PP program can be given an interpretation as logical formula:
\begin{itemize}
\item A fact is just interpreted as a propositional variable $h$.
\item A rule is interpreted as ``body implies head'', in the example: $h \IMPL
  m$. In fact, the symbol \texttt{:-} can be understood as a stylised left
  arrow $\longleftarrow$.
\item The whole program is interpreted as: the facts and rules imply the
  query, thus $(h \AND (h \IMPL m)) \IMPL m$.
\end{itemize}

There are two ways to reason about such a formula:
\begin{itemize}
\item by \emph{forward chaining}: starting from the facts, try to infer more
  facts by applying rules in the direction of the implication. In our case, we
  know $h$. Applying this knowledge to rule $h \IMPL m$\footnote{In logic,
    this rule is called \emph{modus ponens}:
    \url{https://en.wikipedia.org/wiki/Modus_ponens}}, we obtain $m$, and this
  is what we wanted to know in the query.
\item by \emph{backward chaining}: starting from the \emph{goal} we want to
  solve (initially: the query), we try to find out what would be required to
  satisfy the goal, by applying rules in the opposite direction, thus possibly
  generating new goals to be solved, until we arrive at a fact. In our case,
  the goal is $m$. Applying $h \IMPL m$ backwards, we obtain the new goal $h$
  which we know to be true because it is a fact.
\end{itemize}


% ......................................................................
\subsection{Propositional Prolog in Haskell}\label{sec:prolog_in_haskell}

We now give a more formal account of the structure of a PP program. In
extension to what has been said above, the body of a rule can be composed of
several propositions \texttt{p1 \dots pn} but can only have one head
\texttt{h}, so the general format of a rule is \texttt{h :- p1, ..., pn}. The
body is interpreted conjunctively, so the formula corresponding to this rule
is $p_1 \AND \dots \AND p_n \IMPL h$\footnote{Conjunction binds more strongly
  than implication; in parenthesised form, this is $(p_1 \AND \dots \AND p_n) \IMPL h$.}.
A ``conjunction with no elements'' (so for $n=0$) can be taken to be the
constant true $\TRUE$, and in this case, $\TRUE \IMPL h$ is the same as the
fact $h$. For these reasons, we can take facts to be special rules with an
empty body.  Also a goal can consist of several propositions.

\begin{exo}\label{exo:pp_prog_structure}
  We will henceforth represent propositional variables as strings.
  Define the following data types:
  \begin{itemize}
  \item \texttt{Rule} with constructor \texttt{Rl} consisting of a variable
    (head) and a list of variables (body).    
  \item \texttt{Goal} with constructor \texttt{Gl} consisting of a list of
    variables .
  \item \texttt{Prog} with constructor \texttt{Pr} consisting of a list of
    rules and a goal.
  \end{itemize}
\end{exo}

For example, the Socrates program is represented as
\begin{lstlisting}
mortalSocrates :: Prog
mortalSocrates
  = Pr
  [Rl "h" [],
   Rl "m" ["h"]
  ]
  (Gl ["m"])
\end{lstlisting}

A more complex program is the following:

\begin{lstlisting}
abcdProg :: Prog
abcdProg
  = Pr
  [Rl "a" [],
   Rl "d" ["b", "c"],
   Rl "d" ["a", "c"],
   Rl "c" ["a"]
  ]
  (Gl ["d", "c"])
\end{lstlisting}
corresponding to:

\begin{alltt}
  a.
  d :- b, c.
  d :- a, c.
  c :- a.
  ?- d, c.
\end{alltt}

  
% ......................................................................
\subsection{Propositional Prolog as Formula}\label{sec:prolog_as_formula}

\begin{exo}\label{exo:pp_prog_to_form}
Write functions \texttt{ruleToForm}, \texttt{goalToForm}, \texttt{progToForm}
that translate rules, goals and programs to formulas as outlined above. The
translation of a goal \texttt{?- g1, \dots gn} is the conjunction $g_1 \AND
\dots g_n$.

\emph{Hint:} it is useful to have auxiliary functions
\begin{itemize}
\item \texttt{implies :: Form -> Form -> Form} representing implication, where
  $A \IMPL B$ is equivalent to $(\NOT A) \OR B$. You do not have to modify the
  type \texttt{Form}!
\item \texttt{conj :: [Form] -> Form} constructs a conjunction of the formulas
  in the list, where \texttt{conj []} is \texttt{(C True)}. The function is
  easy to write with \texttt{foldl}. 
\end{itemize}
\end{exo}

With the functions of \secref{sec:evaluating_models}, you now have a first
handle to check whether a query succeeds or not: convert the Prolog program to
a formula and then check whether it is valid:

\begin{lstlisting}
> valid (progToForm mortalSocrates)
True
\end{lstlisting}

Let's call the following ``false'' program \texttt{immortalSocrates}:
\begin{alltt}
  h.
  h :- m.
  ?- m
\end{alltt}

\begin{lstlisting}
> valid (progToForm immortalSocrates)
False
> allModels (Not (progToForm immortalSocrates))
[[("m",False),("h",True)]]
\end{lstlisting}
The counter-model evidencing that the program's formula is \emph{not} correct
shows that both preconditions are true ($h$ is true, and $m \IMPL h$ is true
because $m$ is false), and $m$ is false and thus cannot be a conclusion of the
preconditions.


% ......................................................................
\subsection{Interlude: some list functions}\label{sec:list_functions}

Before proceeding to the PP interpreter, let's recall some list processing functions.

\begin{exo}\label{exo:any_all}
  Make sure you are familiar with the functions \texttt{any} (checking whether
  there is \emph{some} element with a property in a list) and \texttt{all}
  (checking whether \emph{all} elements have the property). Thus,
  \begin{itemize}
  \item how do you check whether there is a number divisible by 3 in the list
    \texttt{[1, 4, 5, 6]}?
  \item how do you check whether all numbers in this list are divisible by 3?
  \end{itemize}
\end{exo}

List comprehension (see chapter 9.7 in the Haskell book) is a concise way of
filtering (parts of) elements out of a list that have a specific property.
\begin{exo}\label{exo:list_comprehension}\
  \begin{itemize}
  \item Use list comprehension to compute the squares of the negative numbers
    of \texttt{[1, -3, 4, -5]} (the result should be \texttt{[9,25]}).
  \item Use list comprehension to find the rules that have a given head, for
    example the rules \texttt{d :- b, c.} and \texttt{d :- a, c} for head
    \texttt{d} in \texttt{abcdProg}.
  \end{itemize}
\end{exo}

% ......................................................................
\subsection{Running Propositional Prolog Programs}\label{sec:running_prolog}

A second approach to check a PP program is to run it the way the Prolog engine
would do. This can be much more efficient than by conversion to a formula, but
there are also some downsides as we will see shortly.

Let's see how Prolog would execute \texttt{abcdProg} by backward chaining. The
goal is initially \texttt{d, c}, and both \texttt{d} and \texttt{c} have to be
derivable.
\begin{itemize}
\item Let's first try to solve \texttt{d}, for which we have two rule
  candidates:
  \begin{itemize}
  \item let's first try with \texttt{d :- b, c.} In order for \texttt{d} to be
    derivable, both \texttt{b} and \texttt{c} have to, but we see that there
    is no way to show \texttt{b} (no rule has \texttt{b} as head), so this
    rule fails.
  \item let's now try with \texttt{d :- a, c}, which generates 
    \texttt{a, c}, both of which have to be satisfied. The subgoal \texttt{a}
    can immediately seen to hold (as a fact), and also \texttt{c} is seen to
    hold via rule \texttt{c :- a} and fact \texttt{a}. 
  \end{itemize}
  Altogether, we conclude that \texttt{d} is derivable.
\item We now try to show \texttt{c}. Indeed, we have already shown it in a
  previous subgoal, but Prolog is not intelligent enough to remember it and so
  does the whole job again.
\end{itemize}

Let us summarise the procedure:
\begin{itemize}
\item In order to \emph{solve a goal} (consisting of one or more propositions, such
  as \texttt{d, c}), we have to solve \emph{all} the propositions of the goal.
\item In order to \emph{solve a proposition} \texttt{p}, we have to find \emph{some} rule
  \texttt{h :- p1, \dots pn} in the rule set whose head matches the
  proposition (\emph{i.e.} \texttt{h == p}), and then solve the goal \texttt{p1, \dots pn}.
\end{itemize}

This characterization gives rise to two mutually recursive procedures:

\begin{exo}\label{sec:solve_prolog} \
  \begin{itemize}
  \item Write function \texttt{solveGoal :: [Rule] -> [String] -> Bool} such
    that \texttt{solveGoal rls ps} solves all the propositions in
    \texttt{ps}. This function will call \texttt{solveProp}. Use the Haskell
    function \texttt{all} to express that all propositions have to be solved.
  \item Write function \texttt{solveProp :: [Rule] -> String -> Bool} such
    that \texttt{solveProp rls p} solves proposition \texttt{p} using rules
    \texttt{rls}. This function will call \texttt{solveGoal}. Use the Haskell
    function \texttt{any} to express that some matching rule among
    \texttt{rls} does the job. Also see exercise~\ref{exo:list_comprehension}.
  \end{itemize}
  In spite of the lengthy description, each of these functions is just one
  line long.
\end{exo}


\begin{exo}\label{sec:runProg}
It is now easy to define function \texttt{runProg :: Prog -> Bool} that
fetches the rule set and goal (list of propositions) out of a program and
starts \texttt{solveGoal} on them.

\noindent
\emph{Example:}
\begin{lstlisting}
> runProg mortalSocrates
True
> runProg immortalSocrates
False
\end{lstlisting}
\end{exo}


What is the difference between the method of checking a program as in
\secref{sec:prolog_as_formula} and running it as described here? The latter
method can be dramatically faster, but can lead to looping behaviour, whereas
the model finding approach always finds an answer (terminates), at least in
principle. We illustrate the problem with the following program:

\begin{lstlisting}
nonterminating :: Prog
nonterminating
  = Pr
  [Rl "h" [],
   Rl "m" ["m", "h"],
   Rl "m" ["h"]
  ]
  (Gl ["m"])
\end{lstlisting}
When trying to solve \texttt{m}, we look through the rules from top to
bottom and first find \texttt{m :- m, h.}, so we choose it and now have to
solve the goal \texttt{m, h} corresponding to its body, which amounts to
solving the proposition \texttt{m} first, and so we are back to the original
goal. From a logical perspective, the formula $m \AND h \IMPL m$ corresponding
to \texttt{m :- m, h.} is vacuously true and thus could be deleted without
altering the meaning of the program, but operationally, the rule leads to
non-termination.

We can avoid the problem by changing the rule order:
\begin{lstlisting}
terminating :: Prog
terminating
  = Pr
  [Rl "h" [],
   Rl "m" ["h"],
   Rl "m" ["m", "h"]
  ]
  (Gl ["m"])
\end{lstlisting}
This program yields the desired result without looping - just try it out!

\end{document}






%%% Local Variables: 
%%% mode: latex
%%% TeX-master: t
%%% coding: utf-8-unix
%%% End: 
