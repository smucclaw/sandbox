\documentclass{beamer}


%%%%%%%%% PACKAGES %%%%%%%%%%%%%%%%%
\usepackage[utf8]{inputenc}
\usepackage[T1]{fontenc}
\usepackage{amsmath}
\usepackage{amssymb}
\usepackage{amsfonts}
\usepackage{alltt}
\renewcommand{\ttdefault}{txtt} % to resolve a problem with bold fonts in alltt

\usepackage{listings}
\lstset{language=haskell,basicstyle=\ttfamily}

\usepackage{proof}
\inferLineSkip=4pt  % increase spacing between lines; default is 2pt

\usepackage[table]{xcolor}
\usepackage[colorlinks,hyperindex,bookmarks,linkcolor=blue,citecolor=blue,urlcolor=blue]{hyperref}
\usepackage{todonotes}

\usepackage{fancyhdr}

%\usepackage{floatflt}
\usepackage{wrapfig}
\usepackage{caption}
\usepackage{subcaption}
\usepackage{framed}
\usepackage{multicol}

\makeatletter

\usepackage{babel}
\usepackage{graphicx}
\usepackage{theorem}

\usepackage{tikz}
\usetikzlibrary{trees}
\usetikzlibrary{positioning} 
\usepackage{tikzsymbols}


\makeatother

%%%%%%%%% GEOMETRY %%%%%%%%%%%%%%%%%
\addtolength{\topmargin}{-15mm}
\addtolength{\textheight}{25mm}
\addtolength{\oddsidemargin}{-20mm}
\setlength{\textwidth}{16cm}

%%%%%%%%% DECLS / DEFNS %%%%%%%%%%%%%%%%%

\usepackage{comment}
\specialcomment{solution}
{\todo[inline]{BEGIN SOLUTION}}
{\todo[inline]{END SOLUTION}}

{\theorembodyfont{\rmfamily} 
  \newtheorem{exo}{Exercise}
  \newtheorem{rem}{Remark}
}

% Macros for references
\newcommand{\polyref}[1]{polycopié, {#1}}

%%%%%%%%% END DECLS / DEFNS %%%%%%%%%%%%%%%%%

%%% Local Variables: 
%%% mode: latex
%%% coding: utf-8-unix
%%% End: 

%======================================================================
%%%%%%%%% DECLS / DEFNS %%%%%%%%%%%%%%%%%

% redefine existing green color
\newcommand{\brightgreen}[1]{{\color{green}#1}}
\definecolor{darkgreen}{rgb}{0,0.7,0}
\newcommand{\green}[1]{{\color{darkgreen}#1}}

\newcommand{\blue}[1]{{\color{blue}#1}}
%\newcommand{\green}[1]{{\color{green}#1}}
\newcommand{\red}[1]{{\color{red}#1}}
\newcommand{\yellow}[1]{{\color{yellow}#1}}
\definecolor{grey}{rgb}{0.5,0.5,0.5}
\newcommand{\grey}[1]{{\color{grey}#1}}
\newcommand{\black}[1]{{\color{black}#1}}

% Logique
\newcommand{\IMPL}[0]{\longrightarrow}
\newcommand{\IMPLL}[0]{\Longrightarrow} % another implication, to make
                                % a difference with reduction relations
\newcommand{\AND}[0]{\land}
\newcommand{\OR}[0]{\lor}
\newcommand{\NOT}[0]{\lnot}
\newcommand{\FALSE}[0]{\perp}
\newcommand{\TRUE}[0]{\top}
\newcommand{\IFF}[0]{\leftrightarrow}
\newcommand{\BIGAND}[1]{\bigwedge_{#1}}
\newcommand{\BIGOR}[1]{\bigvee_{#1}}
\newcommand{\BIGANDC}[2]{\bigwedge_{#1|#2}} % bigand with constraint
\newcommand{\BIGORC}[2]{\bigvee_{#1|#2}}    % bigor with constraint
\newcommand{\ufb}[0]{\mbox{$\stackrel{?}{=}$}} % unifiable

% semantique
\newcommand{\semtrans}[3]{\mbox{$\langle #1, #2 \rangle \rightarrow #3$}}
\newcommand{\statesel}[2]{\mbox{$#1 . #2$}}
\newcommand{\stateupd}[3]{\mbox{$#1 . #2 \leftarrow #3$}}


% Macros for mathematical notation

\def\N{{\mathbf N}}                      % ensemble des entiers naturels
\def\Z{{\mathbf Z}}                      % ensemble des entiers relatifs
\newcommand{\zmod}[1]{\mbox{$ \Z/{#1} \Z$}}  % anneau Z mod m


% WP-calcul
\newcommand{\wpform}[1]{\mbox{\{\( #1 \)\} }}
\newcommand{\subst}[3]{ #1 [ #2 \mbox{$\leftarrow\ $} #3]}
\newcommand{\pfp}[2]{\mbox{\emph{pfp}(\texttt{#1}, \mbox{$#2$})}}
\newcommand{\evb}[1]{\emph{évaluable}($#1$)}

% special references
\newcommand{\secref}[1]{\S~\ref{#1}}
\newcommand{\transpref}[1]{transparent, p.~\ref{#1}}


\newtheorem{theorem}{Théorème}
\newtheorem{lemma}{Lemme}
\newtheorem{definition}{Définition}
\newtheorem{logic}{Principe}


%%% Local Variables: 
%%% mode: latex
%%% TeX-master: "main"
%%% coding: utf-8
%%% End: 


\title{Typing}

\author{Martin Strecker}
\date{2021-01-15}


%======================================================================

\begin{document}


%======================================================================

\begin{frame}
  \titlepage
\end{frame}



%======================================================================
\section{Typing -- setting the stage}

%-------------------------------------------------------------
\begin{frame}[fragile]\frametitle{Motivation: What are types good for?}

  \blue{Observation:} Natural language is ``typed'':
  \begin{itemize}
  \item \emph{e.g.:} Boolean / numeric:
    \begin{itemize}
    \item \emph{Should we meet this evening?} (Boolean)
    \item \emph{At which time should we meet?} (Numeric)
    \end{itemize}
  \item \emph{e.g.:} Animate / inanimate:
    \begin{itemize}
    \item \emph{What should we eat?}
    \item \emph{Whom should we eat?}
    \end{itemize}
  \end{itemize}

\end{frame}

%-------------------------------------------------------------
\begin{frame}[fragile]\frametitle{Motivation: Purposes of typing}

  \begin{itemize}
  \item High-level: as a \blue{first sanity check} to prevent manifest errors:
    \begin{itemize}
    \item In C:  \texttt{if (x = 1) ... } (assignment, numeric)\\
      instead of: \texttt{if (x == 1) ...} (Boolean condition)
    \item In Caml: What do you want?
      \begin{itemize}
      \item \texttt{1.2 +. 3.4} (result is \texttt{4.6}) or
      \item \texttt{int\_of\_float(1.2) + int\_of\_float(3.4)} (result is \texttt{4})
      \end{itemize}
    \end{itemize}

  \item Low-level: during \blue{compilation} to determine memory layout etc.
    \begin{itemize}
    \item \texttt{int} and \texttt{float} usually mapped to different kinds of registers
    \end{itemize}
  \end{itemize}
  
\end{frame}

%-------------------------------------------------------------
\begin{frame}[fragile]\frametitle{Different dimensions of typing}

  \begin{itemize}
  \item \blue{Dynamic vs. static:}
    \begin{itemize}
    \item Dynamic: the type of an expression is only known during its
      execution
    \item Static: the type is already known at compile time
    \end{itemize}
  \item \blue{Strong vs. weak}
    \begin{itemize}
    \item Strong: typing discipline is strictly imposed
    \item Weak: deviations from typing discipline are tolerated
    \end{itemize}
  \item \blue{Unique vs. multiple}
    \begin{itemize}
    \item Unique: an expression has only one type
    \item Multiple: an expression can have several types
    \end{itemize}
  \end{itemize}

\end{frame}

%-------------------------------------------------------------
\begin{frame}[fragile]\frametitle{Dynamic vs. static typing}

  \blue{Static} for example in Haskell. Impossible to define:
  \begin{lstlisting}
foo n = if (1 / n) > 0 then 1 else True

<interactive>:3:29: error:
Could not deduce (Num Bool) arising from literal 1
  \end{lstlisting}
  
  \vspace{2mm}
  \blue{Dynamic} for example in Python.
  \vspace{-5mm}
  \begin{columns}[t]
    \column{.5\textwidth}
  \begin{lstlisting}[language=python]
def foo(n):
    if ((1 / n) > 0):
        return 1
    else:
        return True
      \end{lstlisting}
      \column{.5\textwidth}
      \begin{alltt}
>>> foo(2)
1
>>> foo(-2)
True
>>> foo(0)
ZeroDivisionError
>>> foo([])
TypeError
      \end{alltt}
    \end{columns}

\end{frame}

%-------------------------------------------------------------
\begin{frame}[fragile]\frametitle{Unique vs. multiple typing}

  Most object-oriented languages have \blue{multiple} typing\\
  \emph{Example:} an instance can be of class \texttt{Employee} and of class
  \texttt{Person}

  \vspace{5mm}
  At first glance, it seems that Haskell admits multiple typing:
  \begin{itemize}
  \item \texttt{length [1, 2, 3]}
  \item \texttt{length [True, False]}
  \end{itemize}

  But there is only one \emph{principle type}, so Haskell has \blue{unique}
  typing:
  \begin{lstlisting}
> :t length
length :: Foldable t => t a -> Int
\end{lstlisting}

\end{frame}

%-------------------------------------------------------------
\begin{frame}[fragile]\frametitle{Ingredients of typing}

  Typing an \blue{expression} $e$ yields a \blue{type} $T$:

  \texttt{3 + 4 : Int}\\
  \texttt{(3 + 4) < 9 : Bool}

  Preliminary format of typing judgement:
  \begin{center}
    $e : T$
  \end{center}

  \pause
  The expression  may contain variables\\
  and its type depend on a \blue{context}:
  \begin{itemize}
  \item if \texttt{x} is of type \texttt{Int}, then \texttt{3 + x : Int}:\\
    \texttt{x: Int} $\vdash$ \texttt{3 + x : Int}
  \item if \texttt{x} is not of numeric type, then \texttt{3 + x} is ill-typed
  \end{itemize}

  \vspace{2mm}
  Full format of typing judgement ($\Gamma$: context; $e$: expression; $T$: type):
  \begin{center}
    $\Gamma \vdash e : T$
  \end{center}
  

\end{frame}

%-------------------------------------------------------------
\begin{frame}[fragile]\frametitle{Interesting problems related to typing}

  \begin{itemize}
  \item \blue{Type checking:} Given $\Gamma$, $e$ and $T$, check that $\Gamma \vdash e : T$

  \item \blue{Type computation:} Given $\Gamma$ and $e$, compute $T$:\\
    $\Gamma \vdash e : \red{?}$ \hspace{4mm} $\leadsto$ \hspace{4mm} $\Gamma \vdash e : T$

    
  \item \blue{Type inference:} Given $e$ (and possibly $T$), compute $\Gamma$:\\
    $\red{?} \vdash e : T$ \hspace{4mm} $\leadsto$ \hspace{4mm} $\Gamma \vdash e : T$\\
    or:\\
    $\red{?} \vdash e : \red{?}$ \hspace{4mm} $\leadsto$ \hspace{4mm} $\Gamma \vdash e : T$

  \item \blue{Realization:} Given $\Gamma$ and $T$, compute an appopriate $e$:\\
    $\Gamma \vdash \red{?} : T$ \hspace{4mm} $\leadsto$ \hspace{4mm} $\Gamma \vdash e : T$
  \end{itemize}

\end{frame}


%======================================================================
\section{Type computation and checking}

%-------------------------------------------------------------
\subsection{Type computation without contextual information}
% -------------------------------------------------------------

%-------------------------------------------------------------
\begin{frame}[fragile]\frametitle{The relation between type computation and checking}

  In most cases:\\
  if we can do type \emph{computation}, we can do type \emph{checking}\\
  (check $\Gamma \vdash e : T$ for given $\Gamma, e, T$):

  \begin{enumerate}
  \item  Given $\Gamma$ and $e$, compute a type $T'$ such that $\Gamma
    \vdash e : T'$
  \item Check that $T$ and $T'$ are \emph{compatible} 
  \end{enumerate}

  We here assume \emph{unique} typing, and ``compatible'' to mean ``the same''.

  \vspace{4mm}
  \emph{Example:} check whether $\vdash (3 + 4) < 5 : Int$.
  \begin{enumerate}
  \item Compute the type of $(3 + 4) < 5$ as $Bool$
  \item $Int \neq Bool$, so the typing relation is not satisfied.
  \end{enumerate}
  
\end{frame}

%-------------------------------------------------------------
\begin{frame}[fragile]\frametitle{Type computation}

Essential idea: Synthesize type information ``from the leaves upwards'' in the
syntax tree.


\emph{Example:}

\begin{onlyenv}<1-3>
Expression $(3 + 4) < (5 - 6)$
\end{onlyenv}
\begin{onlyenv}<4>
Expression $(3 + 4) < (5 - 6)$ \red{: Bool}
\end{onlyenv}

\begin{center}
\begin{onlyenv}<1>
\tikz[level 1/.style={sibling distance=3cm}, level 2/.style={sibling distance=1.5cm} ]
\node {$<$} % root
child { node {$+$}
  child { node {3} }
  child { node {4} }
}
child { node {$-$}
  child { node {5} }
  child { node {6} }
};
\end{onlyenv}
\begin{onlyenv}<2>
\tikz[level 1/.style={sibling distance=3cm}, level 2/.style={sibling distance=1.5cm} ]
\node {$<$} % root
child { node {$+$}
  child { node {3 \red{: Int}} }
  child { node {4 \red{: Int}} }
}
child { node {$-$}
  child { node {5 \red{: Int}} }
  child { node {6 \red{: Int}} }
};
\end{onlyenv}
\begin{onlyenv}<3>
\tikz[level 1/.style={sibling distance=3cm}, level 2/.style={sibling distance=1.5cm} ]
\node {$<$} % root
child { node {$+$ \red{: Int}}
  child { node {3 \red{: Int}} }
  child { node {4 \red{: Int}} }
}
child { node {$-$ \red{: Int}}
  child { node {5 \red{: Int}} }
  child { node {6 \red{: Int}} }
};
\end{onlyenv}
\begin{onlyenv}<4>
\tikz[level 1/.style={sibling distance=3cm}, level 2/.style={sibling distance=1.5cm} ]
\node {$<$ \red{: Bool}} % root
child { node {$+$ \red{: Int}}
  child { node {3 \red{: Int}} }
  child { node {4 \red{: Int}} }
}
child { node {$-$ \red{: Int}}
  child { node {5 \red{: Int}} }
  child { node {6 \red{: Int}} }
};
\end{onlyenv}

\end{center}

\end{frame}


%-------------------------------------------------------------
\begin{frame}[fragile]\frametitle{Type computation}


\emph{Example:}

\begin{onlyenv}<1-3>
Expression $(3 < 4) < (5 - 6)$
\end{onlyenv}
\begin{onlyenv}<4>
Expression $(3 < 4) < (5 - 6)$ \red{ill-typed}
\end{onlyenv}

\begin{center}
\begin{onlyenv}<1>
\tikz[level 1/.style={sibling distance=3cm}, level 2/.style={sibling distance=1.5cm} ]
\node {$<$} % root
child { node {$<$}
  child { node {3} }
  child { node {4} }
}
child { node {$-$}
  child { node {5} }
  child { node {6} }
};
\end{onlyenv}
\begin{onlyenv}<2>
\tikz[level 1/.style={sibling distance=3cm}, level 2/.style={sibling distance=1.5cm} ]
\node {$<$} % root
child { node {$<$}
  child { node {3 \red{: Int}} }
  child { node {4 \red{: Int}} }
}
child { node {$-$}
  child { node {5 \red{: Int}} }
  child { node {6 \red{: Int}} }
};
\end{onlyenv}
\begin{onlyenv}<3>
\tikz[level 1/.style={sibling distance=3cm}, level 2/.style={sibling distance=1.5cm} ]
\node {$<$} % root
child { node {$<$ \red{: Bool}}
  child { node {3 \red{: Int}} }
  child { node {4 \red{: Int}} }
}
child { node {$-$ \red{: Int}}
  child { node {5 \red{: Int}} }
  child { node {6 \red{: Int}} }
};
\end{onlyenv}
\begin{onlyenv}<4>
\tikz[level 1/.style={sibling distance=3cm}, level 2/.style={sibling distance=1.5cm} ]
\node {$<$ \red{\faFlash}} % root
child { node {$<$ \red{: Bool}}
  child { node {3 \red{: Int}} }
  child { node {4 \red{: Int}} }
}
child { node {$-$ \red{: Int}}
  child { node {5 \red{: Int}} }
  child { node {6 \red{: Int}} }
};
\end{onlyenv}

\end{center}

\end{frame}


%-------------------------------------------------------------
\begin{frame}[fragile]\frametitle{Typing rules}

  How to express which operators require which type? 

  \begin{itemize}
  \item Informally:
    \begin{itemize}
    \item The addition ($+$) of two $Int$ gives an $Int$
    \item The comparison (with $<$) of two $Int$ gives a $Bool$
    \end{itemize}

  \item A bit more formally:
    \begin{itemize}
    \item If $e_1$ and $e_2$ are of type $Int$, then $e_1 + e_2$ is of type $Int$
    \item If $e_1$ and $e_2$ are of type $Int$, then $e_1 < e_2$ is of type $Bool$
    \end{itemize}
    
  \item Mathematically:
    $$
\infer[(R+)]{\Gamma \vdash e_1 + e_2 : Int}{
  \Gamma \vdash e_1 : Int & \Gamma \vdash e_2 : Int}
\qquad
\infer[(R<)]{\Gamma \vdash e_1 < e_2 : Bool}{
  \Gamma \vdash e_1 : Int & \Gamma \vdash e_2 : Int}
$$
Note: $\Gamma$ not modified
\end{itemize}

\red{Exercise:} Develop a rule $(R-)$ for subtraction!

\end{frame}


%-------------------------------------------------------------
\begin{frame}[fragile]\frametitle{Typing derivations}

  Successive applications of typing rules give typing derivations.

  In a \blue{type checking mode}, verify that each rule has been applied
  correctly:

$$
  \infer[\blue{(3)}]{\vdash (3 + 4) < (5 - 6) : Bool}{
    \infer[\blue{(1)}]{\vdash 3 + 4 : Int}{
      \vdash 3 : Int & \vdash 4 : Int}
    &&
    \infer[\blue{(2)}]{\vdash 5 - 6 : Int}{
      \vdash 5 : Int & \vdash 6 : Int}
  }
$$

Instances used:
\begin{itemize}
\item for \blue{(1)}: $e_1 = 3$ and $e_2 = 4$   of $(R+)$
\item for \blue{(2)}: $e_1 = 5$ and $e_2 = 6$   of $(R-)$
\item for \blue{(3)}: $e_1 = 3 + 4$ and $e_2 = 5 - 6$  of $(R<)$
\end{itemize}

\end{frame}


%-------------------------------------------------------------
\begin{frame}[fragile]\frametitle{Typing derivations}

  Compare the derivation tree and syntax tree:

\begin{columns}[t]
\column{.45\textwidth}
\small
\vspace{10mm}
$$
  \infer{\vdash (3 + 4) < (5 - 6) : Bool}{
    \infer{\vdash 3 + 4 : Int}{
      \vdash 3 : Int & \vdash 4 : Int}
    &&
    \infer{\vdash 5 - 6 : Int}{
      \vdash 5 : Int & \vdash 6 : Int}
  }
$$
\normalsize
\column{.45\textwidth}
\small
\begin{center}
\tikz[level 1/.style={sibling distance=2.5cm}, level 2/.style={sibling distance=1.2cm} ]
\node {$<$ \red{: Bool}} % root
child { node {$+$ \red{: Int}}
  child { node {3 \red{: Int}} }
  child { node {4 \red{: Int}} }
}
child { node {$-$ \red{: Int}}
  child { node {5 \red{: Int}} }
  child { node {6 \red{: Int}} }
};
\end{center}
\normalsize
\end{columns}

\end{frame}

%-------------------------------------------------------------
\begin{frame}[fragile]\frametitle{Typing derivations}

  In a \blue{type computation mode}, unfold expression, then synthesize type:

\begin{onlyenv}<1>
  $$
  \infer{\vdash (3 + 4) < (5 - 6) : ???}{}
  $$
\end{onlyenv}
\begin{onlyenv}<2>
  $$
  \infer{\vdash (3 + 4) < (5 - 6) : ???}{
    \infer{\vdash 3 + 4 : ???}{}
    &&
    \infer{\vdash 5 - 6 : ???}{}
  }
  $$
\end{onlyenv}
\begin{onlyenv}<3>
  $$
  \infer{\vdash (3 + 4) < (5 - 6) : ???}{
    \infer{\vdash 3 + 4 : ???}{
      \vdash 3 : ??? & \vdash 4 : ???}
    &&
    \infer{\vdash 5 - 6 : ???}{
      \vdash 5 : ??? & \vdash 6 : ???}
  }
  $$
\end{onlyenv}
\begin{onlyenv}<4>
  $$
  \infer{\vdash (3 + 4) < (5 - 6) : ???}{
    \infer{\vdash 3 + 4 : ???}{
      \vdash 3 : Int & \vdash 4 : Int}
    &&
    \infer{\vdash 5 - 6 : ???}{
      \vdash 5 : Int & \vdash 6 : Int}
  }
  $$
\end{onlyenv}
\begin{onlyenv}<5>
  $$
  \infer{\vdash (3 + 4) < (5 - 6) : ???}{
    \infer{\vdash 3 + 4 : Int}{
      \vdash 3 : Int & \vdash 4 : Int}
    &&
    \infer{\vdash 5 - 6 : Int}{
      \vdash 5 : Int & \vdash 6 : Int}
  }
  $$
\end{onlyenv}
\begin{onlyenv}<6>
  $$
  \infer{\vdash (3 + 4) < (5 - 6) : Bool}{
    \infer{\vdash 3 + 4 : Int}{
      \vdash 3 : Int & \vdash 4 : Int}
    &&
    \infer{\vdash 5 - 6 : Int}{
      \vdash 5 : Int & \vdash 6 : Int}
  }
  $$
\end{onlyenv}

\end{frame}

%-------------------------------------------------------------
\begin{frame}[fragile]\frametitle{Type errors derivations}

  What goes wrong with ill-typed expressions?

  \vspace{3mm}
  \emph{Example:} $(3 < 4) < (5 - 6)$

    $$
  \infer{\vdash (3 < 4) < (5 - 6) : ???}{
    \infer{\vdash 3 < 4 : Bool}{
      \vdash 3 : Int & \vdash 4 : Int}
    &&
    \infer{\vdash 5 - 6 : Int}{
      \vdash 5 : Int & \vdash 6 : Int}
  }
  $$

  \vspace{3mm}
  \begin{itemize}
  \item No rule is available to complete the deriviation $\leadsto$ type error
  \item Note: \emph{inductively} defined rule set: for making a derivation,
    you
    \begin{itemize}
    \item may use all the rules of the set
    \item and only these rules
    \end{itemize}
  \end{itemize}

\end{frame}


%-------------------------------------------------------------
\subsection{Type computation with a context}
% -------------------------------------------------------------


%-------------------------------------------------------------
\begin{frame}[fragile]\frametitle{Observation}

  So far, our expressions consist of:
  \begin{itemize}
  \item constants (leaves of syntax tree)
  \item operators ($+, -, <$ \dots) applied to expressions
  \end{itemize}
  In the typing rules, $\Gamma$ remains unused.

  Currently:
  \begin{itemize}
  \item no variables
  \item no functions
  \item consequently, no function application
  \end{itemize}
  

\end{frame}


%-------------------------------------------------------------
\begin{frame}[fragile]\frametitle{What is the context good for?} 

  When binding an expression to a name, you progressively build up a context:
\vspace{-3mm}
\begin{columns}[t]
\column{.45\textwidth}
\begin{verbatim}
Prelude> x = 4
Prelude> y = 5
Prelude> :t x
x :: Num p => p
Prelude> b = True
Prelude> :t b
b :: Bool
\end{verbatim}
\column{.45\textwidth}
\begin{verbatim}
Prelude> :t (x + y)
(x + y) :: Num a => a
Prelude> :t (x + z)
interactive:1:6: error: 
Variable not in scope: z
\end{verbatim}
\end{columns}

\vspace{3mm}
Leaving aside type classes, after declaring \texttt{x, y, z} we get the
context: $x : Int, y: Int, b : Bool$, and thus:

\begin{itemize}
\item $x : Int, y: Int, b : Bool \vdash x + y: Int$
\item $x : Int, y: Int, b : Bool \vdash x + z:$ \faFlash
\end{itemize}

\end{frame}


%-------------------------------------------------------------
\begin{frame}[fragile]\frametitle{Typing expressions with variables}

  We now allow \blue{variables} in expressions and\\
  extend the rule set by the following \blue{variable rule}:

  $$
  \infer[(Var)]{\Gamma \vdash x : T}{
    lookup(\Gamma, x) = T}
  $$
  Here:
  \begin{itemize}
  \item $lookup$ traverses $\Gamma$ from right to left;
  \item retrieves the type associated with the first occurrence of $x$
  \item (error if $x$ not in $\Gamma$)
  \end{itemize}
\end{frame}

%-------------------------------------------------------------
\begin{frame}[fragile]\frametitle{Typing expressions with variables}

  The order of the variables in $\Gamma$ is important:\\
  $lookup(x: Int, y: Int, x: Bool, x) = Bool$

\begin{alltt}
Prelude> x = 4
Prelude> y = 5
Prelude> x = True
Prelude> :t x
x :: Bool
\end{alltt}
The later declaration \emph{shadows} the earlier.

\end{frame}


%-------------------------------------------------------------
\begin{frame}[fragile]\frametitle{Variable declarations in functions}

Variable declarations build up a \emph{local context}, in analogy to the
\emph{global context} seen before:

\begin{verbatim}
Prelude> :t (\x y -> x + y)
(\x y -> x + y) :: Num a => a -> a -> a
Prelude> :t (\x y -> x + z)
interactive:1:14: error: Variable not in scope: z
\end{verbatim}

\end{frame}


%-------------------------------------------------------------
\begin{frame}[fragile]\frametitle{Typing functions}

  We now extend expressions by:
  \begin{itemize}
  \item \blue{Abstractions} \verb|\x -> e| of variable \texttt{x} over
    expression \texttt{e}
  \item \blue{Applications} \texttt{f a} of function \texttt{f} to argument \texttt{a}
  \end{itemize}

  Syntactic conventions:
  \begin{itemize}
  \item Abstractions with several variables correspond to \emph{iterated}
    abstractions (right associativity):\\
    \verb|\x y z -> e| \hspace{3mm} = \hspace{3mm} \verb|\x -> (\y -> (\z -> e))|
  \item Applications with several arguments correspond to \emph{iterated}
    applications (left associativity):\\
    \texttt{f a b c}\hspace{3mm} = \hspace{3mm} \texttt{(((f a) b) c)}
  \end{itemize}
  
\end{frame}


%-------------------------------------------------------------
\begin{frame}[fragile]\frametitle{Typing functions}

  Let us assume (counterfactual!) that we can annotate variables with types:
    \verb|\(x ::Int) -> x + 1|

    \vspace{3mm}
    \blue{Typing rule for abstractions:}
      $$
  \infer[(Abs)]{\Gamma \vdash \backslash (x :: A) \to e \;\; : \;\; A \to B}{
    \Gamma, x: A \vdash e : B}
  $$

    \vspace{3mm}
    \blue{Typing rule for applications:}
      $$
  \infer[(App)]{\Gamma \vdash (f \; a) \; : \;  B}{
    \Gamma \vdash f : A \to B & \Gamma \vdash a : A}
  $$
\end{frame}


%-------------------------------------------------------------
\begin{frame}[fragile]\frametitle{Typing functions}

  \emph{Example:} Let us verify:

\begin{verbatim}
Prelude> :t (\x -> \y -> x + y) (3 :: Int)
(\x -> \y -> x + y) (3 :: Int) :: Int -> Int
\end{verbatim}

\small
  $$
  \infer[(App)]{\vdash  \blue{(\backslash (x :: Int) \to \backslash (y :: Int) \to x + y)\; 3} \; : \; \red{Int \to Int}}{
    \infer[(Abs)]{\vdash  \blue{\backslash (x :: Int) \to \backslash (y :: Int) \to x + y}\; : \; \red{Int \to Int \to Int}}{
      \infer[(Abs)]{x : Int \vdash \blue{\backslash (y :: Int) \to x + y} \; : \; \red{Int \to Int}}{
        \infer[(R+)]{x : Int, y : Int  \vdash \blue{x + y} \; : \; \red{Int}}{
          x : Int, y : Int  \vdash \blue{x} : \red{Int}
          &
          x : Int, y : Int  \vdash \blue{y} : \red{Int}
        }}}
    &
    \vdash \blue{3} : \red{Int}
    }
  $$
\normalsize

\end{frame}


%-------------------------------------------------------------
\begin{frame}[fragile]\frametitle{Properties of typing}

  Typing is to predict correctly what kind of result you will get when
  evaluating an expression.

  \emph{Ex.:} Assume expression \verb|(\x y -> (x * y) / (x + y)) | has type
  \texttt{Flot -> Float -> Float}\\
  (predicted \emph{statically} by the type checker)

\vspace{3mm}
  When applied to two \texttt{Float} and \emph{evaluating} the expression
  \begin{itemize}
  \item the result is guaranteed to be a \texttt{Float}
  \item and not a \texttt{Bool} or a list \dots
  \item and not to produce a dynamic type error
  \item (but may produce other runtime errors, \emph{e.g.} division by zero)
  \end{itemize}
  Slogan: \emph{well-typed programs don't go wrong}

\end{frame}

%-------------------------------------------------------------
\begin{frame}[fragile]\frametitle{Type soundness}

  More formally: Define an \blue{evaluation} relation $e \twoheadrightarrow v$
  between an expression $e$ and value $v$.

  \vspace{2mm}
  \emph{NB:} The evaluation relation defines a \blue{semantics} of expressions.

  \vspace{2mm}
  \emph{Ex.:} $(\backslash x\; y \to (x * y) / (x + y))\; 2.8\; 4.2
  \twoheadrightarrow 1.68$

  \vspace{4mm}
  \blue{Subject reduction:}
  \begin{itemize}
  \item If $\Gamma \vdash e : T$
  \item and $e \twoheadrightarrow v$
  \item then $\Gamma \vdash v : T$
  \end{itemize}
  

\end{frame}

%-------------------------------------------------------------
\begin{frame}[fragile]\frametitle{Type soundness}

  What can defeat type soundness?\\
  An inappropriate definition of evaluation, e.g.: \blue{dynamic scoping}
  
  \emph{Example:} a convoluted Haskell expression:
  \begin{verbatim}
Prelude> (\x -> (\y x -> y < 2 && x) (x + 1) True) 3
False
  \end{verbatim}
\vspace{-4mm}  
  \red{Wrong evaluation:}
\verb|(\x -> (\y x -> y < 2 && x) (x + 1) True) 3| $\twoheadrightarrow$
\verb|(\x -> (\x -> (x + 1) < 2 && x) True) 3| $\twoheadrightarrow$
\verb|(\x -> (True + 1) < 2 && True) 3| $\twoheadrightarrow$ 
\verb|(True + 1) < 2 && True| $\twoheadrightarrow$ \faFlash

\vspace{2mm}
  \green{Right evaluation:} rename bound variables
\verb|(\x -> (\y x -> y < 2 && x) (x + 1) True) 3| $\twoheadrightarrow$
\verb|(\x -> (\b -> (x + 1) < 2 && b) True) 3| $\twoheadrightarrow$ \dots


\end{frame}

%-------------------------------------------------------------
\subsection{Implementation}
% -------------------------------------------------------------



%======================================================================
\section{Type inference}



%======================================================================
\section{An ignorant's introduction to type theory}






%-------------------------------------------------------------

\end{document}


%%% Local Variables: 
%%% mode: latex
%%% TeX-master: t
%%% coding: utf-8
%%% End: 
