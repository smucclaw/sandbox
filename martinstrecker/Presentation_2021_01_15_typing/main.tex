\documentclass{beamer}


%%%%%%%%% PACKAGES %%%%%%%%%%%%%%%%%
\usepackage[utf8]{inputenc}
\usepackage[T1]{fontenc}
\usepackage{amsmath}
\usepackage{amssymb}
\usepackage{amsfonts}
\usepackage{alltt}
\renewcommand{\ttdefault}{txtt} % to resolve a problem with bold fonts in alltt

\usepackage{listings}
\lstset{language=haskell,basicstyle=\ttfamily}

\usepackage{proof}
\inferLineSkip=4pt  % increase spacing between lines; default is 2pt

\usepackage[table]{xcolor}
\usepackage[colorlinks,hyperindex,bookmarks,linkcolor=blue,citecolor=blue,urlcolor=blue]{hyperref}
\usepackage{todonotes}

\usepackage{fancyhdr}

%\usepackage{floatflt}
\usepackage{wrapfig}
\usepackage{caption}
\usepackage{subcaption}
\usepackage{framed}
\usepackage{multicol}

\makeatletter

\usepackage{babel}
\usepackage{graphicx}
\usepackage{theorem}

\usepackage{tikz}
\usetikzlibrary{trees}
\usetikzlibrary{positioning} 
\usepackage{tikzsymbols}


\makeatother

%%%%%%%%% GEOMETRY %%%%%%%%%%%%%%%%%
\addtolength{\topmargin}{-15mm}
\addtolength{\textheight}{25mm}
\addtolength{\oddsidemargin}{-20mm}
\setlength{\textwidth}{16cm}

%%%%%%%%% DECLS / DEFNS %%%%%%%%%%%%%%%%%

\usepackage{comment}
\specialcomment{solution}
{\todo[inline]{BEGIN SOLUTION}}
{\todo[inline]{END SOLUTION}}

{\theorembodyfont{\rmfamily} 
  \newtheorem{exo}{Exercise}
  \newtheorem{rem}{Remark}
}

% Macros for references
\newcommand{\polyref}[1]{polycopié, {#1}}

%%%%%%%%% END DECLS / DEFNS %%%%%%%%%%%%%%%%%

%%% Local Variables: 
%%% mode: latex
%%% coding: utf-8-unix
%%% End: 

%======================================================================
%%%%%%%%% DECLS / DEFNS %%%%%%%%%%%%%%%%%

% redefine existing green color
\newcommand{\brightgreen}[1]{{\color{green}#1}}
\definecolor{darkgreen}{rgb}{0,0.7,0}
\newcommand{\green}[1]{{\color{darkgreen}#1}}

\newcommand{\blue}[1]{{\color{blue}#1}}
%\newcommand{\green}[1]{{\color{green}#1}}
\newcommand{\red}[1]{{\color{red}#1}}
\newcommand{\yellow}[1]{{\color{yellow}#1}}
\definecolor{grey}{rgb}{0.5,0.5,0.5}
\newcommand{\grey}[1]{{\color{grey}#1}}
\newcommand{\black}[1]{{\color{black}#1}}

% Logique
\newcommand{\IMPL}[0]{\longrightarrow}
\newcommand{\IMPLL}[0]{\Longrightarrow} % another implication, to make
                                % a difference with reduction relations
\newcommand{\AND}[0]{\land}
\newcommand{\OR}[0]{\lor}
\newcommand{\NOT}[0]{\lnot}
\newcommand{\FALSE}[0]{\perp}
\newcommand{\TRUE}[0]{\top}
\newcommand{\IFF}[0]{\leftrightarrow}
\newcommand{\BIGAND}[1]{\bigwedge_{#1}}
\newcommand{\BIGOR}[1]{\bigvee_{#1}}
\newcommand{\BIGANDC}[2]{\bigwedge_{#1|#2}} % bigand with constraint
\newcommand{\BIGORC}[2]{\bigvee_{#1|#2}}    % bigor with constraint
\newcommand{\ufb}[0]{\mbox{$\stackrel{?}{=}$}} % unifiable

% semantique
\newcommand{\semtrans}[3]{\mbox{$\langle #1, #2 \rangle \rightarrow #3$}}
\newcommand{\statesel}[2]{\mbox{$#1 . #2$}}
\newcommand{\stateupd}[3]{\mbox{$#1 . #2 \leftarrow #3$}}


% Macros for mathematical notation

\def\N{{\mathbf N}}                      % ensemble des entiers naturels
\def\Z{{\mathbf Z}}                      % ensemble des entiers relatifs
\newcommand{\zmod}[1]{\mbox{$ \Z/{#1} \Z$}}  % anneau Z mod m


% WP-calcul
\newcommand{\wpform}[1]{\mbox{\{\( #1 \)\} }}
\newcommand{\subst}[3]{ #1 [ #2 \mbox{$\leftarrow\ $} #3]}
\newcommand{\pfp}[2]{\mbox{\emph{pfp}(\texttt{#1}, \mbox{$#2$})}}
\newcommand{\evb}[1]{\emph{évaluable}($#1$)}

% special references
\newcommand{\secref}[1]{\S~\ref{#1}}
\newcommand{\transpref}[1]{transparent, p.~\ref{#1}}


\newtheorem{theorem}{Théorème}
\newtheorem{lemma}{Lemme}
\newtheorem{definition}{Définition}
\newtheorem{logic}{Principe}


%%% Local Variables: 
%%% mode: latex
%%% TeX-master: "main"
%%% coding: utf-8
%%% End: 


\title{Typing}

\author{Martin Strecker}
\date{2021-01-15}


%======================================================================

\begin{document}


%======================================================================

\begin{frame}
  \titlepage
\end{frame}



%======================================================================
\section{Typing -- setting the stage}

%-------------------------------------------------------------
\begin{frame}[fragile]\frametitle{Motivation: What are types good for?}

  \blue{Observation:} Natural language is ``typed'':
  \begin{itemize}
  \item \emph{e.g.:} Boolean / numeric:
    \begin{itemize}
    \item \emph{Should we meet this evening?} (Boolean)
    \item \emph{At which time should we meet?} (Numeric)
    \end{itemize}
  \item \emph{e.g.:} Animate / inanimate:
    \begin{itemize}
    \item \emph{What should we eat?}
    \item \emph{Whom should we eat?}
    \end{itemize}
  \end{itemize}
  
\end{frame}

%-------------------------------------------------------------
\begin{frame}[fragile]\frametitle{Motivation: Purposes of typing}

  \begin{itemize}
  \item High-level: as a \blue{first sanity check} to prevent manifest errors:
    \begin{itemize}
    \item In C:  \texttt{if (x = 1) ... } (assignment, numeric)\\
      instead of: \texttt{if (x == 1) ...} (Boolean condition)
    \item In Caml: What do you want?
      \begin{itemize}
      \item \texttt{1.2 +. 3.4} (result is \texttt{4.6}) or
      \item \texttt{int\_of\_float(1.2) + int\_of\_float(3.4)} (result is \texttt{4})
      \end{itemize}
    \end{itemize}

  \item Low-level: during \blue{compilation} to determine memory layout etc.
    \begin{itemize}
    \item \texttt{int} and \texttt{float} usually mapped to different kinds of registers
    \end{itemize}
  \end{itemize}
  
\end{frame}

%-------------------------------------------------------------
\begin{frame}[fragile]\frametitle{Different dimensions of typing}

  \begin{itemize}
  \item \blue{Dynamic vs. static:}
    \begin{itemize}
    \item Dynamic: the type of an expression is only known during its
      execution
    \item Static: the type is already known at compile time
    \end{itemize}
  \item \blue{Strong vs. weak}
    \begin{itemize}
    \item Strong: typing discipline is strictly imposed
    \item Weak: deviations from typing discipline are tolerated
    \end{itemize}
  \item \blue{Unique vs. multiple}
    \begin{itemize}
    \item Unique: an expression has only one type
    \item Multiple: an expression can have several types
    \end{itemize}
  \end{itemize}

\end{frame}

%-------------------------------------------------------------
\begin{frame}[fragile]\frametitle{Dynamic vs. static typing}

  \blue{Static} for example in Haskell. Impossible to define:
  \begin{lstlisting}
foo n = if (1 / n) > 0 then 1 else True

<interactive>:3:29: error:
Could not deduce (Num Bool) arising from literal 1
  \end{lstlisting}
  
  \vspace{2mm}
  \blue{Dynamic} for example in Python.
  \vspace{-5mm}
  \begin{columns}[t]
    \column{.5\textwidth}
  \begin{lstlisting}[language=python]
def foo(n):
    if ((1 / n) > 0):
        return 1
    else:
        return True
      \end{lstlisting}
      \column{.5\textwidth}
      \begin{alltt}
>>> foo(2)
1
>>> foo(-2)
True
>>> foo(0)
ZeroDivisionError
>>> foo([])
TypeError
      \end{alltt}
    \end{columns}

\end{frame}

%-------------------------------------------------------------
\begin{frame}[fragile]\frametitle{Unique vs. multiple typing}

  Most object-oriented languages have \blue{multiple} typing\\
  \emph{Example:} an instance can be of class \texttt{Employee} and of class
  \texttt{Person}

  \vspace{5mm}
  At first glance, it seems that Haskell admits multiple typing:
  \begin{itemize}
  \item \texttt{length [1, 2, 3]}
  \item \texttt{length [True, False]}
  \end{itemize}

  But there is only one \emph{principle type}, so Haskell has \blue{unique}
  typing:
  \begin{lstlisting}
> :t length
length :: Foldable t => t a -> Int
\end{lstlisting}

\end{frame}

%-------------------------------------------------------------
\begin{frame}[fragile]\frametitle{Ingredients of typing}

  Typing an \blue{expression} $e$ yields a \blue{type} $T$:

  \texttt{3 + 4 : Int}\\
  \texttt{(3 + 4) < 9 : Bool}

  Preliminary format of typing judgement:
  \begin{center}
    $e : T$
  \end{center}

  \pause
  The expression  may contain variables\\
  and its type depend on a \blue{context}:
  \begin{itemize}
  \item if \texttt{x} is of type \texttt{Int}, then \texttt{3 + x : Int}:\\
    \texttt{x: Int} $\vdash$ \texttt{3 + x : Int}
  \item if \texttt{x} is not of numeric type, then \texttt{3 + x} is ill-typed
  \end{itemize}

  \vspace{2mm}
  Full format of typing judgement ($\Gamma$: context; $e$: expression; $T$: type):
  \begin{center}
    $\Gamma \vdash e : T$
  \end{center}
  

\end{frame}

%-------------------------------------------------------------
\begin{frame}[fragile]\frametitle{Interesting problems related to typing}

  \begin{itemize}
  \item \blue{Type checking:} Given $\Gamma$, $e$ and $T$, check that $\Gamma \vdash e : T$

  \item \blue{Type computation:} Given $\Gamma$ and $e$, compute $T$:\\
    $\Gamma \vdash e : \red{?}$ \hspace{4mm} $\leadsto$ \hspace{4mm} $\Gamma \vdash e : T$

    
  \item \blue{Type inference:} Given $e$ (and possibly $T$), compute $\Gamma$:\\
    $\red{?} \vdash e : T$ \hspace{4mm} $\leadsto$ \hspace{4mm} $\Gamma \vdash e : T$\\
    or:\\
    $\red{?} \vdash e : \red{?}$ \hspace{4mm} $\leadsto$ \hspace{4mm} $\Gamma \vdash e : T$

  \item \blue{Realization:} Given $\Gamma$ and $T$, compute an appopriate $e$:\\
    $\Gamma \vdash \red{?} : T$ \hspace{4mm} $\leadsto$ \hspace{4mm} $\Gamma \vdash e : T$
  \end{itemize}

\end{frame}


%======================================================================
\section{Type computation and checking}

%-------------------------------------------------------------
\begin{frame}[fragile]\frametitle{The relation between type computation and checking}

  In most cases:\\
  if we can do type \emph{computation}, we can do type \emph{checking}\\
  (check $\Gamma \vdash e : T$ for given $\Gamma, e, T$):

  \begin{enumerate}
  \item  Given $\Gamma$ and $e$, compute a type $T'$ such that $\Gamma
    \vdash e : T'$
  \item Check that $T$ and $T'$ are \emph{compatible} 
  \end{enumerate}

  We here assume \emph{unique} typing, and ``compatible'' to mean ``the same''.

\end{frame}



%======================================================================
\section{Type inference}



%======================================================================
\section{An ignorant's introduction to type theory}






%-------------------------------------------------------------

\end{document}


%%% Local Variables: 
%%% mode: latex
%%% TeX-master: t
%%% coding: utf-8
%%% End: 
