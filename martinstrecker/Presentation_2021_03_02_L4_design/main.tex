\documentclass{beamer}


%%%%%%%%% PACKAGES %%%%%%%%%%%%%%%%%
\usepackage[utf8]{inputenc}
\usepackage[T1]{fontenc}
\usepackage{amsmath}
\usepackage{amssymb}
\usepackage{amsfonts}
\usepackage{alltt}
\renewcommand{\ttdefault}{txtt} % to resolve a problem with bold fonts in alltt

\usepackage{listings}
\lstset{language=haskell,basicstyle=\ttfamily}

\usepackage{proof}
\inferLineSkip=4pt  % increase spacing between lines; default is 2pt

\usepackage[table]{xcolor}
\usepackage[colorlinks,hyperindex,bookmarks,linkcolor=blue,citecolor=blue,urlcolor=blue]{hyperref}
\usepackage{todonotes}

\usepackage{fancyhdr}

%\usepackage{floatflt}
\usepackage{wrapfig}
\usepackage{caption}
\usepackage{subcaption}
\usepackage{framed}
\usepackage{multicol}

\makeatletter

\usepackage{babel}
\usepackage{graphicx}
\usepackage{theorem}

\usepackage{tikz}
\usetikzlibrary{trees}
\usetikzlibrary{positioning} 
\usepackage{tikzsymbols}


\makeatother

%%%%%%%%% GEOMETRY %%%%%%%%%%%%%%%%%
\addtolength{\topmargin}{-15mm}
\addtolength{\textheight}{25mm}
\addtolength{\oddsidemargin}{-20mm}
\setlength{\textwidth}{16cm}

%%%%%%%%% DECLS / DEFNS %%%%%%%%%%%%%%%%%

\usepackage{comment}
\specialcomment{solution}
{\todo[inline]{BEGIN SOLUTION}}
{\todo[inline]{END SOLUTION}}

{\theorembodyfont{\rmfamily} 
  \newtheorem{exo}{Exercise}
  \newtheorem{rem}{Remark}
}

% Macros for references
\newcommand{\polyref}[1]{polycopié, {#1}}

%%%%%%%%% END DECLS / DEFNS %%%%%%%%%%%%%%%%%

%%% Local Variables: 
%%% mode: latex
%%% coding: utf-8-unix
%%% End: 

%======================================================================
%%%%%%%%% DECLS / DEFNS %%%%%%%%%%%%%%%%%

% redefine existing green color
\newcommand{\brightgreen}[1]{{\color{green}#1}}
\definecolor{darkgreen}{rgb}{0,0.7,0}
\newcommand{\green}[1]{{\color{darkgreen}#1}}

\newcommand{\blue}[1]{{\color{blue}#1}}
%\newcommand{\green}[1]{{\color{green}#1}}
\newcommand{\red}[1]{{\color{red}#1}}
\newcommand{\yellow}[1]{{\color{yellow}#1}}
\definecolor{grey}{rgb}{0.5,0.5,0.5}
\newcommand{\grey}[1]{{\color{grey}#1}}
\newcommand{\black}[1]{{\color{black}#1}}

% Logique
\newcommand{\IMPL}[0]{\longrightarrow}
\newcommand{\IMPLL}[0]{\Longrightarrow} % another implication, to make
                                % a difference with reduction relations
\newcommand{\AND}[0]{\land}
\newcommand{\OR}[0]{\lor}
\newcommand{\NOT}[0]{\lnot}
\newcommand{\FALSE}[0]{\perp}
\newcommand{\TRUE}[0]{\top}
\newcommand{\IFF}[0]{\leftrightarrow}
\newcommand{\BIGAND}[1]{\bigwedge_{#1}}
\newcommand{\BIGOR}[1]{\bigvee_{#1}}
\newcommand{\BIGANDC}[2]{\bigwedge_{#1|#2}} % bigand with constraint
\newcommand{\BIGORC}[2]{\bigvee_{#1|#2}}    % bigor with constraint
\newcommand{\ufb}[0]{\mbox{$\stackrel{?}{=}$}} % unifiable

% semantique
\newcommand{\semtrans}[3]{\mbox{$\langle #1, #2 \rangle \rightarrow #3$}}
\newcommand{\statesel}[2]{\mbox{$#1 . #2$}}
\newcommand{\stateupd}[3]{\mbox{$#1 . #2 \leftarrow #3$}}


% Macros for mathematical notation

\def\N{{\mathbf N}}                      % ensemble des entiers naturels
\def\Z{{\mathbf Z}}                      % ensemble des entiers relatifs
\newcommand{\zmod}[1]{\mbox{$ \Z/{#1} \Z$}}  % anneau Z mod m


% WP-calcul
\newcommand{\wpform}[1]{\mbox{\{\( #1 \)\} }}
\newcommand{\subst}[3]{ #1 [ #2 \mbox{$\leftarrow\ $} #3]}
\newcommand{\pfp}[2]{\mbox{\emph{pfp}(\texttt{#1}, \mbox{$#2$})}}
\newcommand{\evb}[1]{\emph{évaluable}($#1$)}

% special references
\newcommand{\secref}[1]{\S~\ref{#1}}
\newcommand{\transpref}[1]{transparent, p.~\ref{#1}}


\newtheorem{theorem}{Théorème}
\newtheorem{lemma}{Lemme}
\newtheorem{definition}{Définition}
\newtheorem{logic}{Principe}


%%% Local Variables: 
%%% mode: latex
%%% TeX-master: "main"
%%% coding: utf-8
%%% End: 


\title{L4 design}

\author{Martin Strecker}
\date{2021-03-02}


%======================================================================

\begin{document}


%======================================================================

\begin{frame}
  \titlepage
\end{frame}



%======================================================================
\section{Snapshot: Baby-L4 now}


%-------------------------------------------------------------
\begin{frame}[fragile]\frametitle{General structure of an L4 file}

  \begin{itemize}
  \item Lexicon
\begin{alltt}
\textbf{lexicon}
Business -> business_2 #from WordNet
Value -> value_1 
\end{alltt}
    
  \item List of class definitions (class $\approx$ type):
\begin{alltt}
\textbf{class} Business \{
      foo: Int
      bar: Bool -> (Int,Int)
\}

\textbf{class} LawRelatedService \textbf{extends} Business \{
\}
\end{alltt}

    
  \end{itemize}

\end{frame}


%-------------------------------------------------------------
\begin{frame}[fragile]\frametitle{General structure of an L4 file}

  \begin{itemize}
  \item Declarations:
\begin{alltt}
\textbf{decl} AssociatedWith: LegalPractitioner -> 
                      Appointment -> Bool
\textbf{decl} MayAcceptApp : LegalPractitioner -> 
                      Appointment -> Bool
\textbf{decl} ProhibitedBusiness : Business -> Bool
\end{alltt}
    
  \end{itemize}

\end{frame}

%-------------------------------------------------------------
\begin{frame}[fragile]\frametitle{General structure of an L4 file}


  \begin{itemize}
  \item Rules:
\begin{alltt}
\textbf{rule} <r1a>
\textbf{for} lpr: LegalPractitioner, app: Appointment
\textbf{if} (exists bsn : Business. 
         AssociatedWithAppB app bsn 
      \&\& IncompatibleDignity bsn)
\textbf{then} MustNotAcceptApp lpr app
\end{alltt}
  
\item Assertions / Goals to be proved:
\begin{alltt}
\textbf{assert} 
  exists lpr: LegalPractitioner. 
  exists app: Appointment. 
      MayAcceptApp lpr app
\end{alltt}
  \end{itemize}


\end{frame}

  
%======================================================================
\section{What to do with Baby-L4}

%-------------------------------------------------------------
\begin{frame}[fragile]\frametitle{Class definitions}

  \blue{Purpose:}
  \begin{itemize}
  \item Separation of data and ``rules''
  \item Used in type checking
  \end{itemize}
  
  \blue{What to do with it?}
  \begin{itemize}
  \item Parse data from data description languages:\\
    YAML, Jason, \dots\\
    \dots and check conformity with class defs
  \item Generate language stubs for OO languages
  \item Use in verification tools like Alloy
  \end{itemize}

\end{frame}

%-------------------------------------------------------------
\begin{frame}[fragile]\frametitle{Semantics of classes and fields}

  \blue{Mathematically:} Class = set of objects

  \blue{Pragmatically:}
  \begin{itemize}
  \item Fields corrsponding to relation declarations?
  \item Fields/ methods in the sense of OO languages?
  \end{itemize}
  Complementary, not incompatible views.

\end{frame}


%-------------------------------------------------------------
\begin{frame}[fragile]\frametitle{Semantics of classes and fields}

  \blue{Fields as function / relation declarations?}

\begin{alltt}
\textbf{class} LegalPractitioner \textbf{extends} Person \{
\}
\textbf{decl} AssociatedWith:
     LegalPractitioner -> Appointment -> Bool
\end{alltt}
  
the same as?:

\begin{alltt}
\textbf{class} LegalPractitioner \textbf{extends} Person \{
    AssociatedWith: Appointment -> Bool
\}
\end{alltt}

Functional / relational view: we write

\begin{alltt}
  forall lpr  : LegalPractitioner.
  exists app: Appointment.
     AssociatedWith lpr app
\end{alltt}

\end{frame}


%-------------------------------------------------------------
\begin{frame}[fragile]\frametitle{Semantics of classes and fields}

  \blue{Fields as components of a record}


  ``Relational'' view of components (as in Alloy)
\begin{alltt}
\textbf{class} LegalPractitioner \textbf{extends} Person \{
    salary: Int
\}
\end{alltt}

Understanding: \texttt{salary} is a relation\\
\texttt{LegalPractitioner} $\times$ \texttt{Int}

and \texttt{lpr.salary} is relation composition.

\end{frame}

%-------------------------------------------------------------
\begin{frame}[fragile]\frametitle{Semantics of classes and fields}

  \blue{Fields as components of a record}


  ``Functional'' view of components:
\begin{alltt}
\textbf{class} LegalPractitioner \textbf{extends} Person \{
      salary: Int
\}
\end{alltt}

For \texttt{lpr: LegalPractitioner}, one can write:\\
\texttt{lpr.salary}

which is syntactic sugar of \texttt{salary(lpr)}

\red{Downside:} under the relational interpretation, \texttt{lpr.salary} is
not uniquely determined $\leadsto$ cardinality annotations.

\end{frame}
  
%======================================================================
\section{Moving to L4}


%-------------------------------------------------------------
\begin{frame}[fragile]\frametitle{Rule modifiers}

  ... such as \texttt{subject to} clauses:

  \begin{itemize}
  \item annotations in rules 
  \item compilation of preconditions
  \end{itemize}


\end{frame}


%-------------------------------------------------------------
\begin{frame}[fragile]\frametitle{Deontic statements}

  Further investigations:

  \begin{itemize}
  \item Obligations as requirements on a module
  \item Permissions as requirements on its environment
  \item Composition of contracts
  \item Rely-Guarantee reaasoning
  \end{itemize}

\end{frame}



%-------------------------------------------------------------

\end{document}


%%% Local Variables: 
%%% mode: latex
%%% TeX-master: t
%%% coding: utf-8
%%% End: 
