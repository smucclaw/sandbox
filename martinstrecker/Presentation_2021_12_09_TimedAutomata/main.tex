\documentclass{beamer}


%%%%%%%%% PACKAGES %%%%%%%%%%%%%%%%%
\usepackage[utf8]{inputenc}
\usepackage[T1]{fontenc}
\usepackage{amsmath}
\usepackage{amssymb}
\usepackage{amsfonts}
\usepackage{alltt}
\renewcommand{\ttdefault}{txtt} % to resolve a problem with bold fonts in alltt

\usepackage{listings}
\lstset{language=haskell,basicstyle=\ttfamily}

\usepackage{proof}
\inferLineSkip=4pt  % increase spacing between lines; default is 2pt

\usepackage[table]{xcolor}
\usepackage[colorlinks,hyperindex,bookmarks,linkcolor=blue,citecolor=blue,urlcolor=blue]{hyperref}
\usepackage{todonotes}

\usepackage{fancyhdr}

%\usepackage{floatflt}
\usepackage{wrapfig}
\usepackage{caption}
\usepackage{subcaption}
\usepackage{framed}
\usepackage{multicol}

\makeatletter

\usepackage{babel}
\usepackage{graphicx}
\usepackage{theorem}

\usepackage{tikz}
\usetikzlibrary{trees}
\usetikzlibrary{positioning} 
\usepackage{tikzsymbols}


\makeatother

%%%%%%%%% GEOMETRY %%%%%%%%%%%%%%%%%
\addtolength{\topmargin}{-15mm}
\addtolength{\textheight}{25mm}
\addtolength{\oddsidemargin}{-20mm}
\setlength{\textwidth}{16cm}

%%%%%%%%% DECLS / DEFNS %%%%%%%%%%%%%%%%%

\usepackage{comment}
\specialcomment{solution}
{\todo[inline]{BEGIN SOLUTION}}
{\todo[inline]{END SOLUTION}}

{\theorembodyfont{\rmfamily} 
  \newtheorem{exo}{Exercise}
  \newtheorem{rem}{Remark}
}

% Macros for references
\newcommand{\polyref}[1]{polycopié, {#1}}

%%%%%%%%% END DECLS / DEFNS %%%%%%%%%%%%%%%%%

%%% Local Variables: 
%%% mode: latex
%%% coding: utf-8-unix
%%% End: 

%======================================================================
%%%%%%%%% DECLS / DEFNS %%%%%%%%%%%%%%%%%

% redefine existing green color
\newcommand{\brightgreen}[1]{{\color{green}#1}}
\definecolor{darkgreen}{rgb}{0,0.7,0}
\newcommand{\green}[1]{{\color{darkgreen}#1}}

\newcommand{\blue}[1]{{\color{blue}#1}}
%\newcommand{\green}[1]{{\color{green}#1}}
\newcommand{\red}[1]{{\color{red}#1}}
\newcommand{\yellow}[1]{{\color{yellow}#1}}
\definecolor{grey}{rgb}{0.5,0.5,0.5}
\newcommand{\grey}[1]{{\color{grey}#1}}
\newcommand{\black}[1]{{\color{black}#1}}

% Logique
\newcommand{\IMPL}[0]{\longrightarrow}
\newcommand{\IMPLL}[0]{\Longrightarrow} % another implication, to make
                                % a difference with reduction relations
\newcommand{\AND}[0]{\land}
\newcommand{\OR}[0]{\lor}
\newcommand{\NOT}[0]{\lnot}
\newcommand{\FALSE}[0]{\perp}
\newcommand{\TRUE}[0]{\top}
\newcommand{\IFF}[0]{\leftrightarrow}
\newcommand{\BIGAND}[1]{\bigwedge_{#1}}
\newcommand{\BIGOR}[1]{\bigvee_{#1}}
\newcommand{\BIGANDC}[2]{\bigwedge_{#1|#2}} % bigand with constraint
\newcommand{\BIGORC}[2]{\bigvee_{#1|#2}}    % bigor with constraint
\newcommand{\ufb}[0]{\mbox{$\stackrel{?}{=}$}} % unifiable

% semantique
\newcommand{\semtrans}[3]{\mbox{$\langle #1, #2 \rangle \rightarrow #3$}}
\newcommand{\statesel}[2]{\mbox{$#1 . #2$}}
\newcommand{\stateupd}[3]{\mbox{$#1 . #2 \leftarrow #3$}}


% Macros for mathematical notation

\def\N{{\mathbf N}}                      % ensemble des entiers naturels
\def\Z{{\mathbf Z}}                      % ensemble des entiers relatifs
\newcommand{\zmod}[1]{\mbox{$ \Z/{#1} \Z$}}  % anneau Z mod m


% WP-calcul
\newcommand{\wpform}[1]{\mbox{\{\( #1 \)\} }}
\newcommand{\subst}[3]{ #1 [ #2 \mbox{$\leftarrow\ $} #3]}
\newcommand{\pfp}[2]{\mbox{\emph{pfp}(\texttt{#1}, \mbox{$#2$})}}
\newcommand{\evb}[1]{\emph{évaluable}($#1$)}

% special references
\newcommand{\secref}[1]{\S~\ref{#1}}
\newcommand{\transpref}[1]{transparent, p.~\ref{#1}}


\newtheorem{theorem}{Théorème}
\newtheorem{lemma}{Lemme}
\newtheorem{definition}{Définition}
\newtheorem{logic}{Principe}


%%% Local Variables: 
%%% mode: latex
%%% TeX-master: "main"
%%% coding: utf-8
%%% End: 


\title{Timed Automata}

\author{Martin Strecker}
\date{2021-06-03}


%======================================================================

\begin{document}


%======================================================================

\begin{frame}
  \titlepage
\end{frame}



%======================================================================
\section{Timed Automata - Basic notions}



%-------------------------------------------------------------
\begin{frame}[fragile]\frametitle{Recap: Classical Finite-State Automata}
  
The chewing gum automaton:

  \begin{center}
    \includegraphics[scale=0.4]{Figures/chewing_gum_automaton.png}
  \end{center}

\end{frame}



%-------------------------------------------------------------
\begin{frame}[fragile]\frametitle{Recap: Classical Finite-State Automata}
  
  \blue{Central ingredients:}
  \begin{itemize}
  \item States (circles)
  \item Transitions (arcs between states) annotated with symbols of an alphabet
  \item One \emph{initial} state, marked with triangle (where execution starts)
  \item One or several \emph{final} states, marked by double circle (where execution ends)
  \end{itemize}


  \blue{Central notions:}
  \begin{itemize}
  \item Symbols of an alphabet represent possible \emph{actions}
  \item Accepted \emph{word} of an automaton: sequence of symbols leading from initial to final state
    \begin{itemize}
    \item \emph{Ex.:}  \texttt{ddgscprcprt} is accepted
    \item \emph{Ex.:} \texttt{dget} is not accepted
    \end{itemize}
    The set of accepted words of an automaton make up the \emph{language} recognized by the automaton. 
  \end{itemize}

\end{frame}


%-------------------------------------------------------------
\begin{frame}[fragile]\frametitle{Towards Timed Automata}

  \blue{What remains:} Notion of \emph{discrete state}\\
    not appropriate for \emph{continuous} actions (\emph{e.g.} ball rolling down a slope)

\vspace{5mm}
\blue{What is new:}
\begin{itemize}
\item Two kinds of transitions:
  \begin{itemize}
  \item \emph{Action} transition $\to^a$, inducing state change
  \item \emph{Delay} transition  $\to^d$, inducing progression of time
  \end{itemize}
\item Notion of timer / \emph{clock}
\item Further annotations on
  \begin{itemize}
  \item states (\emph{invariants})
  \item transitions (\emph{guards})
  \end{itemize}
\item Executions may be infinite
\end{itemize}

\end{frame}

%-------------------------------------------------------------
\begin{frame}[fragile]\frametitle{Using Uppaal}

  \begin{itemize}
  \item Download from \url{https://uppaal.org/downloads/}
  \item Unpack archive
  \item In the resulting repository:
    \begin{itemize}
    \item launch \texttt{./uppaal\&}
    \item \emph{or} run Java jar file: \texttt{java -jar uppaal.jar\&}
    \end{itemize}
  \item On Mac, it may be necessary to authorize file system access (?)
  \item Open project with \texttt{File / Open System}
  \end{itemize}

\end{frame}

%-------------------------------------------------------------
\begin{frame}[fragile]\frametitle{Some hints}

  \blue{File formats:}
  \begin{itemize}
  \item \texttt{*.xml} Full system information, incl.{} graphics and queries
  \item \texttt{*.xta} Human-readable textual format for automata, no graphics
    info, no queries
  \item \texttt{*.q} queries only
  \end{itemize}

  \blue{Edit node / transition info:}
  \begin{itemize}
  \item by double click on element
  \end{itemize}

  \blue{If information is scattered over the canvas:}
  \begin{itemize}
  \item Click on node / arc to see elements associated with it
  \end{itemize}

  \blue{Notable:}
  \begin{itemize}
  \item Initial state with double circle
  \item No final states (see: queries for verification)
  \end{itemize}
  
\end{frame}

%-------------------------------------------------------------
\begin{frame}[fragile]\frametitle{Example automaton}

  \begin{center}
    \includegraphics[scale=0.4]{Figures/system_access_automaton.png}
  \end{center}

  See: \texttt{fv} repository,  \texttt{Experiments/Uppaal/system\_access.xml}
\end{frame}

%-------------------------------------------------------------
\begin{frame}[fragile]\frametitle{Annotations}

  \blue{States} can be annotated with:
  \begin{itemize}
  \item \emph{invariant:} a condition to be satisfied to remain in this state.\\
    \emph{e.g.:} \texttt{clock1 <= 4} of \texttt{awaiting\_access}
  \end{itemize}

  \vspace{3mm}
  \blue{Transitions} can be annotated with:
  \begin{itemize}
  \item \emph{guard:} a condition to be satisfied to take the transition\\
        \emph{e.g.:} \texttt{clock1 >= 3} 
  \item \emph{update:} an assignment to a clock or data variable:
         \texttt{clock1 = 0} 
      \item \emph{synchronisation} of the form \texttt{e!} (send event) or
        \texttt{e?} (receive event)
  \end{itemize}

  \vspace{3mm}
  \emph{Note:} a \blue{condition} is a conjunction of
  \begin{itemize}
  \item clock comparisons: \texttt{clock1 >= 3}
  \item non-temporal Boolean conditions (\texttt{access\_granted})
  \end{itemize}
  
\end{frame}


%-------------------------------------------------------------
\begin{frame}[fragile]\frametitle{Traces}

  A \emph{configuration} consists of:
  \begin{itemize}
  \item a state 
  \item the values of the clocks
  \item the values of data variables
  \end{itemize}

  A \emph{trace} describes a sequence of transitions (action or delay) a TA can
  take. A trace is \emph{valid} if it respects the node and arc conditions.

  \vspace{3mm}
  For config (state, clock0, clock1, access\_granted):
  \begin{itemize}
  \item Example of valid trace:
  \[(start, 0, 0, false) \to^d (start, 2, 2, false) \to^a (aw, 2, 2, false) \to^d \]
  \[(aw, 3, 3, false) \to^a (req, 3, 0, false) \to^d (req, 7, 4, false) \to^a \]
  \[(aw, 7, 4, true) \to^a (entered, 7, 4, true) \]
  \item Example of invalid trace (arc constraint not satisfied):
  \[(start, 0, 0, false)  \to^a (aw, 0, 0, false) \to^a (entered, 0, 0, false)\]
  \end{itemize}
\end{frame}


%-------------------------------------------------------------
\begin{frame}[fragile]\frametitle{Using the simulator}
  
\end{frame}


%-------------------------------------------------------------
\begin{frame}[fragile]\frametitle{Verifying properties}

  Specify system properties with formulas of \blue{CTL} (Computation Tree Logic)

  \vspace{3mm}

  Two kinds of quantifiers:
  \begin{itemize}
  \item \blue{Path quantifiers} over alternative execution paths:
    \begin{itemize}
    \item $A \phi$: for all executions, $\phi$ holds
    \item $E \phi$: there exists an execution such that $\phi$
    \end{itemize}
  \item \blue{State quantifiers} over states on one path:
    \begin{itemize}
    \item $\Box \phi$ (ASCII: \texttt{[]}): for all states on the path, $\phi$ holds
    \item $\Diamond \phi$ (ASCII: \texttt{<>}): there exists a state on the path such that $\phi$
    \end{itemize}
  \end{itemize}
  
  \vspace{3mm}
  In Uppaal:
  \begin{itemize}
  \item \dots{} and in general in CTL:\\
    Quantifiers only come in pairs $A \Box, A \Diamond, E \Box, E \Diamond$
  \item Apart from that, quantifiers cannot be nested.
  \end{itemize}
  
\end{frame}


%-------------------------------------------------------------
\begin{frame}[fragile]\frametitle{Path and state quantifiers, graphically}

  \begin{center}
    \includegraphics[scale=0.4]{Figures/path_and_state_quantifiers.png}
  \end{center}

\end{frame}


%-------------------------------------------------------------
\begin{frame}[fragile]\frametitle{Verifying properties in Uppaal}

  \begin{itemize}
  \item Open \texttt{Verifier} tab
  \item Type in formula (in \texttt{Query})
  \item Run \texttt{Check}, read off \texttt{Status}
  \item For:
    \begin{itemize}
    \item Satisfied existential properties
    \item Not satisfied universal properties
    \end{itemize}
    generate example / counterexample trace with \texttt{Get Trace}
  \item View trace in \texttt{Simulator} tab
  \end{itemize}

  \vspace{3mm}
  \emph{Example:} It is possible to enter the system:\\
  \texttt{E<> Aut.entered\_system}

  \vspace{3mm}
  \emph{Other queries} (possibly modify automaton):
  \begin{itemize}
  \item Sooner or later, one necessarily enters the system
  \item Clock1 always remains lower than 10
  \end{itemize}


\end{frame}





%======================================================================
\section{Timed Automata in Baby L4}


%-------------------------------------------------------------
\begin{frame}[fragile]\frametitle{Textual formats Uppaal / L4}

  \begin{itemize}
  \item Uppaal can be saved in human-readable \texttt{*.xta} format:
    \begin{itemize}
    \item \texttt{Save system as:} \texttt{system\_access.xta}
    \item save queries as \texttt{system\_access.q}
    \end{itemize}
  \item This is almost the L4 input format:
    \begin{itemize}
    \item Enclose system in declaration
    \item Change comments from \texttt{//} (Uppaal) to \texttt{\#} (L4)
    \item Move variable declarations to top level
    \item Rename some constants: \texttt{true} $\leadsto$ \texttt{True}
    \end{itemize}
  \end{itemize}
  
\end{frame}

%-------------------------------------------------------------
\begin{frame}[fragile]\frametitle{Textual format L4}

  \begin{lstlisting}
system Access {
bool access_granted;
process Aut() {
clock clock0, clock1;
state
    start,
    awaiting_access { clock1 <= 4 },
    request_pending { clock1 <= 5 },
    entered_system;
init
    start;
trans
    awaiting_access -> entered_system { guard clock0 == 7 ; },
    request_pending -> awaiting_access { guard clock1 >= 3;  },
    awaiting_access -> request_pending { guard clock1 >= 3; assign clock1 = 0; },
    start -> awaiting_access { }; } }

assert {printUp} True
  \end{lstlisting}
  

\end{frame}

%-------------------------------------------------------------
\begin{frame}[fragile]\frametitle{Work in Progress}

  \begin{itemize}
  \item TYpe checking of automata\\
    (needs generation of elements in the typing context: states, clocks,
    \dots)
  \item Structures and substructures (systems having automata as components)
  \item Model checking of complex systems via SMT 
  \item Difficulty: computation of fixed points
  \item Longer-reaching aim: Push the limits of what is currently possible in
    TA model checking (``parametric'' systems, more complex conditions)
  \end{itemize}

\end{frame}

%======================================================================
\section{Translating to Timed Automata}


%-------------------------------------------------------------
\begin{frame}[fragile]\frametitle{Approach~1: Actor-based}

  \begin{center}
    \includegraphics[scale=0.3]{Figures/actor_based_translation.png}
  \end{center}

\end{frame}


%-------------------------------------------------------------
\begin{frame}[fragile]\frametitle{Approach~2: Rule-based}

  \begin{center}
    \includegraphics[scale=0.5]{Figures/rule_sf_l4.png}
  \end{center}

\end{frame}

%-------------------------------------------------------------
\begin{frame}[fragile]\frametitle{Approach~2: Rule-based}

  \begin{center}
    \includegraphics[scale=0.4]{Figures/rule_based_translation.png}
  \end{center}

\end{frame}


%-------------------------------------------------------------

\end{document}


%%% Local Variables: 
%%% mode: latex
%%% TeX-master: t
%%% coding: utf-8
%%% End: 
